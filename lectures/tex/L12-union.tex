% &latex
\documentclass{article}

 
\usepackage{amsthm}
\newtheorem{definition}{Definition}
\newtheorem{example}{Example}


\usepackage{outlines}
\usepackage{enumitem}
\setenumerate[1]{label=\arabic*.}
\setenumerate[2]{label=\alph*.}
\setenumerate[3]{label=\roman*.}
\setenumerate[4]{label=\alph*.}

\usepackage{graphicx}
\DeclareGraphicsRule{.tif}{png}{.png}{`convert #1 `dirname #1`/`basename #1 .tif`.png}

% 

\usepackage{framed}  % https://latexcolor.com
\usepackage{xcolor}


\usepackage{tcolorbox, graphicx}
\usepackage{colortbl}
\usepackage{textcomp}

% Colors
%\definecolor{anti-flashwhite}{rgb}{0.95, 0.95, 0.96}
%\definecolor{magnolia}{rgb}{0.97, 0.96, 1.0}
\definecolor{antiquewhite}{rgb}{0.96, 0.96, 0.96} % {0.98, 0.92, 0.84}
%\definecolor{shadecolor}{rgb}{0.95,0.975,0.997}
%\definecolor{shadecode}{rgb}{0.91,0.91,0.91}
%\definecolor{red_1}{rgb}{1,0.8,0.8}
%\definecolor{yellow_1}{rgb}{1,0.96,0.63}
%\definecolor{orange}{rgb}{1,0.5,0}
\definecolor{appleGray}{rgb}{0.75,0.75,0.75}
\definecolor{lightGray}{rgb}{0.975,0.975,0.975}
\definecolor{relhead}{rgb}{0.70,0.80,0.90}
\definecolor{borderGray}{rgb}{0.8,0.8,0.8}
%\definecolor{gray}{rgb}{0.975,0.975,0.975}
\definecolor{nearwhite}{rgb}{0.985,0.985,0.985}
%\definecolor{supergray}{cmyk}{0,0,0.04,0}
%\definecolor{stainlessSteel}{cmyk}{0,0,0.02,0.12}
%\definecolor{mygreen}{rgb}{0,0.6,0}
\definecolor{mygray}{rgb}{0.20,0.20,0.20}
\definecolor{mymauve}{rgb}{0.58,0,0.82}
\definecolor{excel}{rgb}{0.94, 0.94, 0.94}



\usepackage{listings}  

 \lstset{ 
  backgroundcolor=\color{white},  	 % background color; e.g., nearwhite you must add \usepackage{color} or \usepackage{xcolor}; should come as last argument
  basicstyle=\footnotesize\ttfamily,        % the size of the fonts that are used for the code
  breakatwhitespace=false,         		    % sets if automatic breaks should only happen at whitespace
  breaklines=true,                 		    % sets automatic line breaking
  framextopmargin=5pt,
  framexleftmargin=5pt, 
  framexbottommargin=5pt,
  framexrightmargin=0pt,
  framesep=0pt,
  captionpos=b,                    			% sets the caption-position to bottom
  commentstyle=\color{mygreen},    		    % comment style
  morecomment=[s]{/*}{*/},
  deletekeywords={...},            			% if want to delete keywords from the given language
  escapeinside={\%*}{*)},          			% if you want to add LaTeX within your code
  extendedchars=true,              			% lets you use non-ASCII characters; for 8-bits encodings only, does not work with UTF-8
  frame=single,	                   			% adds a frame around the code
  keepspaces=false,               			% keeps spaces in text, useful for keeping indentation of code (possibly needs columns=flexible)
  keywordstyle=\color{blue},      		% keyword style blue
  language=SQL,                 			% the language of the code
  morekeywords={ORDER, USE, DELIMITER, CALL, DECLARE, IF, ELSE, ELSEIF, WHILE, DO, LOOP, REPEAT, UNTIL, CURSOR, FOR, HANDLER, OUT, INTO, FROM, RETURNS, RETURN, FUNCTION, SHOW, TRIGGER, TRIGGERS, END$$, EVENT},   % if you want to add more keywords to the set
  numbers=none,                    			% where to put the line-numbers; possible values are (none, left, right)
  numbersep=0pt,                   			% how far the line-numbers are from the code
  numberstyle=\tiny\color{mygray}, 		    % the style that is used for the line-numbers
  rulecolor=\color{appleGray},  % appleGray     % if not set, the frame-color may be changed on line-breaks within not-black text (e.g. comments (green here))
  sensitive=true,
  showspaces=false,                			% show spaces everywhere adding particular underscores; it overrides 'showstringspaces'
  showstringspaces=false,          		    % underline spaces within strings only
  showtabs=false,                  			% show tabs within strings adding particular underscores
  stepnumber=2,                    			% the step between two line-numbers. If it's 1, each line will be numbered
  stringstyle=\color{mymauve},     		    % string literal style
  tabsize=4,	                   			% sets default tabsize to 2 spaces
  title=\lstname,                  			% show the filename of files included with \lstinputlisting; also try caption instead of title
  upquote=true,      % Straight quotes
  belowcaptionskip=0em,
  belowskip=0em
}
 




\begin{document}

%+Title
\title{Combination queries (\texttt{UNION} \&  \texttt{UPDATE})}
\author{DSC 301: Lecture 12}
\date{March 24, 2021} % \today
\maketitle
%-Title

%+Abstract
%\begin{abstract}
%    There is abstract text that you should replace with your own. 
%\end{abstract}
%-Abstract



%+Contents
% \tableofcontents
%-Contents

% ---------------- %
\begin{outline}[enumerate]

\end{outline}
% ---------------- %
\begin{outline}
        
\end{outline}




% --------- Lecture Objectives ------- %
\section*{Lecture Objectives}
\begin{outline}
        \1  Arithmetic functions
        \1  \texttt{DISTINCT} clause
        \1  Aggregate Functions
       
       
        

\end{outline}

% --------------------------------------------------
\hspace{-0.5cm}\rule[-0.101in]{\textwidth}{0.0025in}
% --------------------------------------------------
% 
% 
%\begin{definition}
%asdfasdf
%\end{definition}









% --------   -------- %
\section*{Combined Queries}
 
So far we have written queries that contain a single \texttt{SELECT} statement (well except subqueries) that returned data form one or more tables.  Combined queries is the mechanism that enables performing multiple queries (multiple \texttt{SELECT} statements) and return results as a single result set.  

Use combined queries to
\begin{outline} 
          \1 Return data from different tables in a single result set
          \1 Perform multiple queries on a single table, but return data                 as one set
\end{outline}

  
% --------------------------------------------------
\hspace{-0.5cm}\rule[-0.101in]{\textwidth}{0.0025in}
% --------------------------------------------------
  
  
   
\subsection*{The \texttt{UNION} Operator}
  
 As in set theory, a union operator combines the elements of two sets where the element is in either (or both) sets.  Note however, if in both, only one is listed (not two). In other words, by defaulty, the  \texttt{UNION} operator automatically removes any duplicate rows from the query results.  This default can be changed to include the duplicates (i.e., \texttt{UNION  ALL}).      \\


\noindent \textbf{Rules on \texttt{UNION} operator}
\begin{outline}[enumerate]
        \1 Each query in a \texttt{UNION} must contain the same columns, expressions, or aggregate functions.

        \1 The columns, expressions, and aggregate functions must occur in the same order in each \texttt{SELECT} statement.

        \1 Column datatypes must be compatible (numeric type or date type).
        
        \1 Only one \texttt{ORDER BY} clause can be used and must be placed after the final \texttt{SELECT} statement.  
\end{outline}
  
  
\noindent \textbf{NOTE}: Often  the same results can be produced using multiple \texttt{WHERE} clauses.  However, \texttt{UNION} can simplify these complex query statements.   










 
  \subsection*{UPDATE}
  
  
\begin{lstlisting}[frame=single]  
UPDATE <TableX> 
SET <col1>  = expression1 [, <col2> = expression2, ...]
[WHERE <someColumn> = value];
\end{lstlisting} 


  
  
  
  
  
\begin{lstlisting}[frame=single]  
UPDATE <TableX> 
SET <colName>  = value
WHERE <someColumn> = value;
\end{lstlisting} 


  
  You can SET
  \begin{outline}
  	\1 One column for one row
	\1 One column for multiple rows
	\1 Multiple columns for one row
  \end{outline}
  
  
  
\begin{lstlisting}[frame=single]  
UPDATE <TableX> 
SET <colName>  = value
WHERE <someColumn> = value;
\end{lstlisting} 


  
  
  
\begin{lstlisting}[frame=single]  
UPDATE <TableX> 
SET <colName>  = value
WHERE <someColumn> = value;
\end{lstlisting} 


  
  
%+Bibliography
%\begin{thebibliography}{99}
%\bibitem{Label1} ...
%\bibitem{Label2} ...
%\end{thebibliography}
%-Bibliography



 

\end{document}
