%&latex
\documentclass{article}

 
\usepackage{amsthm}
\newtheorem{definition}{Definition}
\newtheorem{example}{Example}


\usepackage{outlines}
\usepackage{enumitem}
\setenumerate[1]{label=\arabic*.}
\setenumerate[2]{label=\alph*.}
\setenumerate[3]{label=\roman*.}
\setenumerate[4]{label=\alph*.}

\usepackage{graphicx}
\DeclareGraphicsRule{.tif}{png}{.png}{`convert #1 `dirname #1`/`basename #1 .tif`.png}



\usepackage{framed}  % https://latexcolor.com
\usepackage{xcolor}


\usepackage{tcolorbox, graphicx}
\usepackage{colortbl}
\usepackage{textcomp}

% Colors
%\definecolor{anti-flashwhite}{rgb}{0.95, 0.95, 0.96}
%\definecolor{magnolia}{rgb}{0.97, 0.96, 1.0}
\definecolor{antiquewhite}{rgb}{0.96, 0.96, 0.96} % {0.98, 0.92, 0.84}
%\definecolor{shadecolor}{rgb}{0.95,0.975,0.997}
%\definecolor{shadecode}{rgb}{0.91,0.91,0.91}
%\definecolor{red_1}{rgb}{1,0.8,0.8}
%\definecolor{yellow_1}{rgb}{1,0.96,0.63}
%\definecolor{orange}{rgb}{1,0.5,0}
\definecolor{appleGray}{rgb}{0.75,0.75,0.75}
\definecolor{lightGray}{rgb}{0.975,0.975,0.975}
\definecolor{relhead}{rgb}{0.70,0.80,0.90}
\definecolor{borderGray}{rgb}{0.8,0.8,0.8}
%\definecolor{gray}{rgb}{0.975,0.975,0.975}
\definecolor{nearwhite}{rgb}{0.985,0.985,0.985}
%\definecolor{supergray}{cmyk}{0,0,0.04,0}
%\definecolor{stainlessSteel}{cmyk}{0,0,0.02,0.12}
%\definecolor{mygreen}{rgb}{0,0.6,0}
\definecolor{mygray}{rgb}{0.20,0.20,0.20}
\definecolor{mymauve}{rgb}{0.58,0,0.82}
\definecolor{excel}{rgb}{0.94, 0.94, 0.94}



\usepackage{listings}  

 \lstset{ 
  backgroundcolor=\color{white},  	 % background color; e.g., nearwhite you must add \usepackage{color} or \usepackage{xcolor}; should come as last argument
  basicstyle=\footnotesize\ttfamily,        % the size of the fonts that are used for the code
  breakatwhitespace=false,         		    % sets if automatic breaks should only happen at whitespace
  breaklines=true,                 		    % sets automatic line breaking
  framextopmargin=5pt,
  framexleftmargin=5pt, 
  framexbottommargin=5pt,
  framexrightmargin=0pt,
  framesep=0pt,
  captionpos=b,                    			% sets the caption-position to bottom
  commentstyle=\color{mygreen},    		    % comment style
  morecomment=[s]{/*}{*/},
  deletekeywords={...},            			% if want to delete keywords from the given language
  escapeinside={\%*}{*)},          			% if you want to add LaTeX within your code
  extendedchars=true,              			% lets you use non-ASCII characters; for 8-bits encodings only, does not work with UTF-8
  frame=single,	                   			% adds a frame around the code
  keepspaces=false,               			% keeps spaces in text, useful for keeping indentation of code (possibly needs columns=flexible)
  keywordstyle=\color{blue},      		% keyword style blue
  language=SQL,                 			% the language of the code
  morekeywords={ORDER, USE, DELIMITER, CALL, DECLARE, IF, ELSE, ELSEIF, WHILE, DO, LOOP, REPEAT, UNTIL, CURSOR, FOR, HANDLER, OUT, INTO, FROM, RETURNS, RETURN, FUNCTION, SHOW, TRIGGER, TRIGGERS, END$$, EVENT},   % if you want to add more keywords to the set
  numbers=none,                    			% where to put the line-numbers; possible values are (none, left, right)
  numbersep=0pt,                   			% how far the line-numbers are from the code
  numberstyle=\tiny\color{mygray}, 		    % the style that is used for the line-numbers
  rulecolor=\color{appleGray},  % appleGray     % if not set, the frame-color may be changed on line-breaks within not-black text (e.g. comments (green here))
  sensitive=true,
  showspaces=false,                			% show spaces everywhere adding particular underscores; it overrides 'showstringspaces'
  showstringspaces=false,          		    % underline spaces within strings only
  showtabs=false,                  			% show tabs within strings adding particular underscores
  stepnumber=2,                    			% the step between two line-numbers. If it's 1, each line will be numbered
  stringstyle=\color{mymauve},     		    % string literal style
  tabsize=4,	                   			% sets default tabsize to 2 spaces
  title=\lstname,                  			% show the filename of files included with \lstinputlisting; also try caption instead of title
  upquote=true,      % Straight quotes
  belowcaptionskip=0em,
  belowskip=0em
}
 




\begin{document}

%+Title
\title{Subqueries}
\author{DSC 301: Lecture 10}
\date{March 10, 2021} % \today
\maketitle
%-Title

%+Abstract
\begin{abstract}
    Subqueries provide the means to build queries that are difficult or impossible otherwise.  We have already seen several instances where subqueries would have ``come to the rescue."  In particular, ``what percentage of flights arrive late?" and ``how many  destinations has each plane flown?" 
\end{abstract}
%-Abstract



%+Contents
% \tableofcontents
%-Contents

% ---------------- %
\begin{outline}[enumerate]

\end{outline}
% ---------------- %
\begin{outline}
        
\end{outline}




% --------- Lecture Objectives ------- %
\subsection*{Lecture Objectives}
\begin{outline}
        \1 What is a subquery
        \1  Where to places subqueries
        \1 When to use subqueries
        \1 How to use:  
                \2 with the \texttt{IN} operator
                \2 with comparison operators
                \2 with \texttt{ALL, ANY, SOME, EXISTS}
        \1 How to code subqueries in:
                \2 the \texttt{SELECT} clause
                \2 the \texttt{WHERE} clause
                \2 the \texttt{FROM} clause        
                \2 the \texttt{HAVING} clause
        

\end{outline}

% --------------------------------------------------
\hspace{-0.5cm}\rule[-0.101in]{\textwidth}{0.0025in}
% --------------------------------------------------
% 
% 
%\begin{definition}
%asdfasdf
%\end{definition}








 
  
  
  

\subsection*{Subquery Basics}


\begin{outline}
\1 A \texttt{SELECT} statement that's nested within another SQL\ statement
        \2 Can be used within: \texttt{SELECT,INSERT, UPDATE}, and \texttt{DELETE}
statements

% \1 Subqueries can come after \texttt{SELECT, FROM, WHERE}, and \texttt{HAVING} clauses
\1 Subqueries \textbf{cannot} contain \texttt{ORDER BY} clause. 
\1 Subqueries must be contained within parentheses.    

\1 Subqueries can return:
        \2  a single value,  
        \2 a list of values (i.e., single column), or 
        \2 a table of values.  

\1 Subqueries can be nested

\1Subqueries can pass aggregated values to the main query

\1 Subqueries can become complex, use the ``beautifier'' to split over several lines to increase readability

\1 Most subqueries can be coded as JOINS and most JOINS can be coded as subqueries

 \end{outline}

% --------------------------------------------------
\hspace{-0.5cm}\rule[0.101in]{\textwidth}{0.0025in}
% --------------------------------------------------
% 

\noindent Using subqueries comes down to knowing: \textit{when} to use and \textit{where} to place them (depends on what the subquery returns (single value, list of values, or table of values).   






% --------------------------------------------------
\hspace{-0.5cm}\rule[-0.101in]{\textwidth}{0.0025in}
% --------------------------------------------------
    
  
  


\subsection*{Subquery Placement}
\begin{outline}
 
    \1 In a \texttt{WHERE} clause as a search condition
    \1 In a \texttt{HAVING} clause as a search condition 
    \1 In a \texttt{FROM} clause as a ``pseudo-table''
    \1 In a \texttt{SELECT} clause as a ``pseudo-column''
\end{outline}


\subsubsection*{In a \texttt{WHERE} clause} 





\noindent \textbf{Example}: Which destination had the most flights in 2014?






A subquery in the \texttt{WHERE} clause is very common.  For example, find all planes from carriers that fly to ORD (Chicago O'Hara).  Note, this doesn't ask to find only planes from carriers that fly to ORD, that would be:   










 
 
 
 
 
 
 

  
% CODE: 
\begin{lstlisting}[frame=single]  
SELECT distinct(tailnum) FROM Flights WHERE dest='ORD';
\end{lstlisting}                

\noindent Instead we want all planes of the carriers that fly to ORD.  This may include planes that never land in Chicago.\\  

% --------------------------------------------------
% \hspace{-0.5cm}\rule[-0.101in]{\textwidth}{0.0025in}
% --------------------------------------------------
 
  
All flights whose air time is greater than average of all carriers.
% CODE: 
\begin{lstlisting}[frame=single]  
 select fid, carrier, dest, air_time from Flights where air_time>= all(select
avg(air_time) from Flights group by carrier);
\end{lstlisting} 
  

  
  
  

  
% CODE: 
\begin{lstlisting}[frame=single]  
 select * from Flights where air_time = (select max(air_time) from Flights);
\end{lstlisting} 
 


 
 
 
 
 
 
 
 
 
 
  
  
  
% CODE: 
\begin{lstlisting}[frame=single]  
SELECT DISTINCT
    (tailnum)
FROM
    Flights
WHERE
    carrier IN (SELECT 
            carrier
        FROM
            Flights
        WHERE
            dest = 'ORD');
\end{lstlisting}                

  
  
% CODE: 
\begin{lstlisting}[frame=single]  
SELECT 
    dest, distance
FROM
    Flights
WHERE
    air_time > (SELECT 
            AVG(air_time)
        FROM
            Flights);
\end{lstlisting}                
  
  


  
  
  
  
\subsubsection*{Multiple Table Queries}  
  
 \noindent Although subqueries can (obviously) be used to query single tables, more commonly they are used for multiple table queries.  For example, select all customers from West Virginia.  Why is a subquery needed? Answer: The state is not stored in the Customers table, only the zip code.  Therefore, we need to ``look up'' the zip codes from West Virginia using a subquery. 
  
% CODE: 
\begin{lstlisting}[frame=single]   
select * from Products where manufacturer_id in (select manufacturer_id from Manufacturers where url<>'');
\end{lstlisting}   
  
  
  
  
% CODE: 
\begin{lstlisting}[frame=single]  
SELECT C.id 
FROM Customers as C 
WHERE C.zip in 
  (
    SELECT zip 
    FROM Zipcodes 
    WHERE state='WV'
  ); 
\end{lstlisting} 
  
  
  
  
% CODE: 
\begin{lstlisting}[frame=single]    
# Select 100 products that have not yet sold.
SELECT product_id
FROM Products as P
WHERE P.product_id NOT IN 
  (
    SELECT OrderDetails.product 
    FROM OrderDetails
  )
LIMIT 100;
\end{lstlisting}   
  
  






% ----------------------------------- %

\subsection*{In \texttt{FROM} clause}

% CODE: 
\begin{lstlisting}[frame=single]    
select manufacturer from (select * from Manufacturers where url<>'') as M;
\end{lstlisting}   


% CODE: 
\begin{lstlisting}[frame=single]    
select round(avg(X.price),2),max(X.msrp) from (select price, msrp, inventory
from Products where country='China') as X where inventory>0;

\end{lstlisting}   





   

The maximum average time of flights by carrier:

% CODE: 
\begin{lstlisting}[frame=single]  
select max(avg_time) from (select avg(air_time) as avg_time from Flights
group by carrier) as X;
\end{lstlisting} 

  
  
  
  
% CODE: 
\begin{lstlisting}[frame=single]
SELECT 
    carrier, X.avg_time
FROM
    (SELECT 
        carrier, AVG(air_time) AS avg_time
    FROM
        Flights
    GROUP BY carrier) AS X where X.avg_time>200;
\end{lstlisting}    
    
  
  
  
  
% --------------------------------------------------
\hspace{-0.5cm}\rule[-0.101in]{\textwidth}{0.0025in}
% --------------------------------------------------
    
  
  
  
  
  
  
  
  
  
  
  
  
  
  
  
  
  
  
  
  
  
  



% ----------------------------------- %

\subsection*{In \texttt{SELECT} statement}

\texttt{SELECT} subqueries are seldom used.  

 
% CODE: 
\begin{lstlisting}[frame=single]     
# Percentage of flights that arrive late 
SELECT 
    ROUND(100 * (SELECT 
                    COUNT(fid)
                FROM
                    Flights
                WHERE
                    arr_delay > 0) / COUNT(fid),
            0) AS '% Arrive Late'
FROM
    Flights;
\end{lstlisting}   


% --------------------------------------------------
\hspace{-0.5cm}\rule[-0.101in]{\textwidth}{0.0025in}
% --------------------------------------------------
    
    
  
  
%+Bibliography
%\begin{thebibliography}{99}
%\bibitem{Label1} ...
%\bibitem{Label2} ...
%\end{thebibliography}
%-Bibliography



 

\end{document}
