%&latex
\documentclass{article}

\usepackage{amsmath, amssymb} 
\usepackage{amsthm}
\newtheorem{definition}{Definition}

\usepackage{outlines}
\usepackage{enumitem}
\setenumerate[1]{label=\arabic*.}
\setenumerate[2]{label=\alph*.}
\setenumerate[3]{label=\roman*.}
\setenumerate[4]{label=\alph*.}

\usepackage{graphicx}
\DeclareGraphicsRule{.tif}{png}{.png}{`convert #1 `dirname #1`/`basename #1 .tif`.png}


\usepackage{soul}


\usepackage{tcolorbox, graphicx}
\usepackage{xcolor,colortbl}


\usepackage{framed,color}
\definecolor{excel}{rgb}{0.95,0.975,0.997}






\begin{document}

%+Title
\title{Relational Algebra}
\author{DSC 301: Lecture 3}
\date{\today}
\maketitle
%-Title

%+Abstract
\begin{abstract}
   Relational databases are based on the mathematical relation, which is
the primary topic of this lecture.  We begin with elementary set
theory notions, including basic operations. The Cartesian product is of utmost importance as it provides the foundation of relational database theory.  Finally we examine some algebra on relations.        
\end{abstract}
%-Abstract

%+Contents
%  \tableofcontents
%-Contents

% ---------------- %
\begin{outline}[enumerate]

\end{outline}
% ---------------- %
\begin{outline}
        
\end{outline}




% --------- Lecture Objectives ------- %
\section{Lecture Objectives}
\begin{outline}
        \1  Set theory (background)
                \2  Cartesian Product
        \1  Define relation
        \1  unary map
        \1  binary map
        

\end{outline}

%\begin{definition}
%asdfasdf
%\end{definition}








% --------   -------- %
\section{Set Theory }
 
 \begin{definition}
 A set $X$ is a collection of things. For example, $X = \{1,2,3\}$.  
 \end{definition}
 
 
 
  \begin{definition}
The universe of discourse, or \textbf{universal set}, $U$, is  a collection of all objects under consideration.  For example, in first-semester calculus, $U = \mathbb{R}$, the set of all real numbers.  
 \end{definition}
 
 
  \begin{definition}
The set $A$ is a \textbf{subset} of $B$, written $A \subseteq B$ if every element of $A$ is an element of $B$.   
\[
A \subseteq B  \iff  (\forall x) (x \in A \Rightarrow x \in B)
\] 
 \end{definition}
 
\noindent \underline{Example}: If $A = \{1,2\}$ and $B = \{1,2,3, 4\}$,then $A \subseteq B$, since both $1$ and $2$ from $A$ are also in $B$. 
 
 
 
 
 
 
 
 
 \hspace{-0.5cm}\rule[-0.101in]{\textwidth}{0.0025in}
 
 \subsubsection*{Set Operations}
 
 Let $A$ and $B$ be sets.  
 \begin{outline}

        



        \1 The \textbf{union} of $A$ and $B$, written $A \cup B$, is all the elements in $A$ or $B$.
\[ 
     A \cup B = \{ x :  x \in A  \textrm{ or } x \in B \}   
\]

\underline{Example}: $A = \{1,2,3  \}$ and $B = \{3,4\}$, then $A \cup B = \{1, 2, 3, 4\}$. 



 

        \1  The \textbf{intersection} of $A$ and $B$, written $A \cap B$, is elements common to both $A$ and $B$.
\[ 
     A \cap B = \{ x :  x \in A  \textrm{ and } x \in B \}   
\]

\underline{Example}: $A = \{1,2,3  \}$ and $B = \{3,4\}$, then $A \cap B
= \{3 \}$.






        \1  The \textbf{difference} of $A$ and $B$, written $A - B$, is the elements of $A$ but not $B$.
\[ 
     A - B = \{ x :  x \in A  \textrm{ and } x \not\in B \}   
\]

\underline{Example}: $A = \{1,2,3  \}$ and $B = \{3,4\}$, then $A - B
= \{ 1,2 \}$.





        \1 The \textbf{compliment} of $A$ is the set $A^c = U - A$. Example: $U = \{1, 2, 3, 4, 5\}$ and $A = \{2, 5\}$, then $A^c = \{1, 3, 4\}$.  
        
 \end{outline}

% \noindent \textbf{Note} 1: These are binary set operations.  \\
\textbf{Note}: Operations can be combined, e.g., $A\cap (B \cup C)$.
  
  
  
  
  

%\rule[0.01in]{\textwidth}{0.0025in}
 \hspace{-0.5cm}\rule[-0.101in]{\textwidth}{0.0025in}
% ---------------------------------------------------- % 


 \subsection*{Cartesian Product}
 \begin{definition}
        
Let $A_1$ and $A_2$ be sets.  The set 
\[
A_1 \times A_2 = \{(a_1, a_2) \; | \; a_1 \in A_1 \wedge a_2 \in A_2 \} 
\]
is the \textbf{Cartesian product} of $A_1$ and $A_2$.  

 \end{definition} 
  
  
  


\begin{example}
If $A_1 = \{1, 2, 3\}$ and $A_2 = \{ 3,4\}$, then, 

\[
A_1 \times A_2 = \{(1,3), (1,4), (2,3), (2,4), (3,3),(3,4) \}.
\]
\end{example}

\noindent \textbf{Note}: The Cartesian product can be generalized to the product of  $n$ sets, i.e., 
\[ 
A_1 \times A_2 \times \dots \times A_n.
\]   
$^*$ The letter $A$ is chosen to represent ``attribute".




 
\hspace{-0.5cm}\rule[0.01in]{\textwidth}{0.0025in}
% ---------------------------------------------------- % 
 
  
  
  
  
  
 \begin{definition}[]
 A \textbf{relation} is a subset of a Cartesian product.  Relations are said to be degree $n$.  A relation of degree $1$ is called unary, degree $2$ binary, degree $3$ ternary, and degree $n$, $n$-ary.    
 \end{definition} 
  
 
 \noindent \textbf{Recall}:  - Relations provide the framework for relational databases.   
 
 
  % NOTE: First number in rule is vertical spacing, second width of line,
  % third is thickness of line
\hspace{-0.5cm}\rule[0.01in]{\textwidth}{0.0025in}
% ---------------------------------------------------- % 
   
  
  
  
  
  
  
  
 % -------- Relational Algebra -------- %
\section{Relational Algebra } 
  
  What is algebra?  Are there more than one type of algebra?
  
  \hspace{-0.5cm}\rule[0.01in]{\textwidth}{0.0025in}
% ---------------------------------------------------- % 
  \begin{definition}
        A (mathematical) structure consisting of a set with operations that follow (obey) some rules (or constraints).\\

\noindent  - For example: the set could be the integers, operations addition and multiplication, and rules associative, commutative, distributive, etc.  \\
 
\noindent - Perhaps the most important rule is \textbf{closure}.  

  \end{definition}
  
  \hspace{-0.5cm}\rule[0.01in]{\textwidth}{0.0025in}
% ---------------------------------------------------- % 
  
  \begin{definition}
   A \textbf{relational algebra} defines a set of operations on relations, paralleling usual algebraic operations such as addition, subtraction, or multiplication which operate on numbers.  
       % A \textbf{relational algebra} is an algebra on relations.    
  \end{definition}
  
  \hspace{-0.5cm}\rule[0.01in]{\textwidth}{0.0025in}
% ---------------------------------------------------- % 
  
  
  % Unary Operators
  \begin{outline}
  
       \1 \textbf{Rename}: $\rho_S(R)$ or $\rho_R_{(A_1, A_2, \dots, A_n)}(R)$, where  we rename relation $R$ to $S$ and the attributes of $R$ to $A_1, A_2, \dots, A_n$ respectively.  
       
       
       

 
\AtBeginEnvironment{tabular}{\sffamily}
\begin{table}[h!]
\caption{Simple Relation R}
\begin{center}

R = \begin{tabular}{|c|c|c|c|}
   \hline
  \cellcolor{excel}{$A_1$}  & \cellcolor{excel}{$A_2$}  &   \cellcolor{excel}{$A_3$}
&  \cellcolor{excel}{$A_4$}  \\
  \hline
  1 & 1 & 1  &   1 \\
      \hline
  2 & 1 & 3  &   1 \\
      \hline
  3 & 2 & 3  &   4 \\
      \hline
  4 & 1 & 3  &   1 \\
      \hline
   
\end{tabular}
\end{center}
\label{tab:flatfile}
\end{table}%

\begin{example}
Example: $\rho_S_{(A,B,C,D)}(R)$



\AtBeginEnvironment{tabular}{\sffamily}
\begin{table}[h!]
\caption{Rename relation R to S and its attributes}
\begin{center}

S = \begin{tabular}{|c|c|c|c|}
   \hline
  \cellcolor{excel}{$A$}  & \cellcolor{excel}{$B$}  &   \cellcolor{excel}{$C$}
&  \cellcolor{excel}{$D$}  \\
  \hline
  1 & 1 & 1  &   1 \\
      \hline
  2 & 1 & 3  &   1 \\
      \hline
  3 & 2 & 3  &   4 \\
      \hline
  4 & 1 & 3  &   1 \\
      \hline
   
\end{tabular}
\end{center}
\label{tab:flatfile}
\end{table}%

\end{example}

       
       
       
       
       
       
       
       
       
       
       
       % -------- SELECT ---------- %
       \1 \textbf{Select}: $\sigma_p(R)$
                \2 $p$ is a selection predicate  and $R$ is a relation
                \2 Defined as:
                \[
                        \sigma_p(R) = \{ t : t \in R \wedge p(t) \textrm{ is true} \}
                \]
                \2 Select output rows  (filtering operation)
                \2 Subset of rows                
                
          
           \begin{example}
           
           \AtBeginEnvironment{tabular}{\sffamily}
           
           \begin{table}[!htb]
    \caption{Select: $\sigma_{A_1 >2}(R)$ }
    \begin{minipage}{.5\linewidth}
      % \caption{}
      \centering
     R =    \begin{tabular}{|c|c|c|c|}
   \hline
  \cellcolor{excel}{$A_1$}  & \cellcolor{excel}{$A_2$}  &   \cellcolor{excel}{$A_3$}
&  \cellcolor{excel}{$A_4$}  \\
  \hline
  1 & 1 & 1  &   1 \\
      \hline
  2 & 1 & 3  &   1 \\
      \hline
  3 & 2 & 3  &   4 \\
      \hline
  4 & 1 & 3  &   1 \\
      \hline
   
\end{tabular}
    \end{minipage}%
    \begin{minipage}{.5\linewidth}
      \centering
       %  \caption{}
      $\sigma_{A_1 >2}(R)$ =    \begin{tabular}{|c|c|c|c|}
   \hline
  \cellcolor{excel}{$A_1$}  & \cellcolor{excel}{$A_2$}  &   \cellcolor{excel}{$A_3$}
&  \cellcolor{excel}{$A_4$}  \\
  \hline
  3 & 2 & 3  &   4 \\
      \hline
  4 & 1 & 3  &   1 \\
      \hline
   
\end{tabular}
    \end{minipage} 
\end{table}

\end{example}

       
       
       
       
       
       
       
       
       
       
       
       
       
       
       
       
           
           
           
           
           
           
           
           
           
                
        % -------- Projection ---------- %         
                                
        \1 \textbf{Projection}: $\Pi_{A_1, A_2, \dots, A_k}(R)$
                \2 A Subset of columns

                 \2 Removes duplicate tuples from the output
                        \3 Note how duplicate tuples could arises
        
        
        
        
           \begin{example}
           
           \AtBeginEnvironment{tabular}{\sffamily}
           
           \begin{table}[!htb]
    \caption{Projection: $\Pi_{A_2, A_3}(R)$ }
    \begin{minipage}{.5\linewidth}
      % \caption{}
      \centering
     R =    \begin{tabular}{|c|c|c|c|}
   \hline
  \cellcolor{excel}{$A_1$}  & \cellcolor{excel}{$A_2$}  &   \cellcolor{excel}{$A_3$}
&  \cellcolor{excel}{$A_4$}  \\
  \hline
  1 & 1 & 1  &   1 \\
      \hline
  2 & 1 & 3  &   1 \\
      \hline
  3 & 2 & 3  &   4 \\
      \hline
  4 & 1 & 3  &   1 \\
      \hline
   
\end{tabular}
    \end{minipage}%
    \begin{minipage}{.5\linewidth}
      \centering
       %  \caption{}
     $\Pi_{A_2, A_3}(R)$ =    \begin{tabular}{|c|c|}
   \hline
   \cellcolor{excel}{$A_2$}  &   \cellcolor{excel}{$A_3$}  \\
  \hline
   1 & 1 \\
      \hline
  1 & 3  \\
      \hline
  2 & 3  \\
      \hline
 \st{ 1 } & \st{ 3 }   \\
      \hline
   
\end{tabular}
    \end{minipage} 
\end{table}



 


\end{example}
        
        
        
        
        
        
        
        
        
        
        
        
        
         % -------- Cartesian product ---------- % 
        \1 \textbf{Cartesian product}: $\times$
        
        
        
        
         % -------- Union ---------- % 
        \1 \textbf{Union}: of tuples $\cup$
        
        
        
         % -------- Intersection ---------- % 
        \1 \textbf{Intersection}: of tuples $\cap$
        
        
        % -------- Difference ---------- % 
        \1 \textbf{Set difference}: $-$ %  $\backslash$ 
        
        
        
       
  \end{outline}
  
  
  
  
  Binary Operators
  \begin{outline}
        
  \end{outline}
  
  
  


\begin{definition}
A \textbf{relational database} is a collection of distinct normalized relations.
\end{definition}




The relational model is based on a mathematical relation (see Chapter \ref{chap:theory}).



% A \textbf{table}, the physical representation of a relation, is two-dimensional
arrays consisting of \textit{rows} and \textit{columns} and form the basis
of the relational database model. Rows and columns are also referred to as
\textit{records} and \textit{fields}, or \textit{tuples} and \textit{attributes}.
 Table \ref{tab:altterms}  summarizes the different terminology associated
with the relational model.  

\begin{table}[h!]
    \centering
    \begin{tabular}{lll}
        \toprule
        Relation      &   Table     &   File \\  
       % \midrule 
        Tuple         &   Row        &  Record   \\
       % \midrule 
        Attribute      &   Column     &  Field   \\   
          \bottomrule
    \end{tabular}
    \caption{Synonyms for relational model terms. Terms in each row are equivalent.}
    \label{tab:altterms}
\end{table}
 
 
 
 
 
  

 
\AtBeginEnvironment{tabular}{\sffamily}
\begin{table}[h!]
\caption{Product Orders Spreadsheet.  Note data redunancy and inconsistencies}
\begin{center}

\begin{tabular}{|c|c|c|c|c|c|}
   \hline
  \cellcolor{excel}{Date}  & \cellcolor{excel}{Customer} & \cellcolor{excel}{Address}
&
  \cellcolor{excel}{Product} & 
  \cellcolor{excel}{Qty} & 
  \cellcolor{excel}{Price} \\
  \hline
  12/19/2020 & Tom Kenny & 192 Hill St  & Web Camera  & 1 & \$25\\
      \hline
   12/22/2020 & Jake Angeli & 307 Neil Ave.  &  Buffalo Hat & 1 & \$99\\
  \hline
    12/22/2020 & Jake Angeli & 307 Neil Ave  &  Bird Feeder & 1 & \$50\\
    \hline
     12/23/2020 & Marge Simpson & 742 Evergreen   & Headphones  & 2 & \$70\\
     \hline
     12/23/2020 & Nate Brown & 7851 Park Avenue  & Bird Feeder  & 1 & \$50\\
     \hline
     12/24/2020 & Nate Brown & 7851 Park Ave  & Head phones  & 2 & \$35\\
     \hline
     12/24/2020 & Brian Kenny & 30 Baxter Blvd  & Leather Whip  & 1 & \$21\\
      \hline
\end{tabular}
\end{center}
\label{tab:flatfile}
\end{table}%










% The relational algebra uses set union, set difference, and Cartesian product
from set theory, but adds additional constraints to these operators.


A \textbf{relation} on a (finite or infinite) set $X$ is a subset $R$ of
$X \times X$.  The element $x \in X$ is related to $y \in X$ is denoted as
 $(x,y) \in R$ or $xRy$.  For example, if $X = \mathbb{N}$  then $\le$ is
binary relation. In this case, $(2,3)$ is an element of this relation.  An
$n$-ary relation over $X$ is a subset of $X^n = X \times X \times \dots 
\times X$ (i.e., $X$ crossed with itself $n$ times).   










 
\AtBeginEnvironment{tabular}{\sffamily}
\begin{table}[h!]
\caption{Simple Relation}
\begin{center}

\begin{tabular}{|c|c|c|c|}
   \hline
  \cellcolor{excel}{$A_1$}  & \cellcolor{excel}{$A_2$}  &   \cellcolor{excel}{$A_3$}
&  \cellcolor{excel}{$A_4$}  \\
  \hline
  1 & 1 & 1  &   1 \\
      \hline
  2 & 1 & 3  &   1 \\
      \hline
  3 & 2 & 3  &   4 \\
      \hline
  4 & 1 & 3  &   1 \\
      \hline
   
\end{tabular}
\end{center}
\label{tab:flatfile}
\end{table}%



%+Bibliography
%\begin{thebibliography}{99}
%\bibitem{Label1} ...
%\bibitem{Label2} ...
%\end{thebibliography}
%-Bibliography



 

\end{document}
