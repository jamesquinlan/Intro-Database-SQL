%&latex
\documentclass{article}

 
\usepackage{amsthm}
\newtheorem{definition}{Definition}
\newtheorem{example}{Example}

\usepackage{outlines}
\usepackage{enumitem}
\usepackage{soul}

\setenumerate[1]{label=\arabic*.}
\setenumerate[2]{label=\alph*.}
\setenumerate[3]{label=\roman*.}
\setenumerate[4]{label=\alph*.}

\usepackage{graphicx}
\DeclareGraphicsRule{.tif}{png}{.png}{`convert #1 `dirname #1`/`basename #1 .tif`.png}


\usepackage{framed}  % https://latexcolor.com
\usepackage{xcolor}


\usepackage{tcolorbox, graphicx}
\usepackage{colortbl}
\usepackage{textcomp}

% Colors
%\definecolor{anti-flashwhite}{rgb}{0.95, 0.95, 0.96}
%\definecolor{magnolia}{rgb}{0.97, 0.96, 1.0}
\definecolor{antiquewhite}{rgb}{0.96, 0.96, 0.96} % {0.98, 0.92, 0.84}
%\definecolor{shadecolor}{rgb}{0.95,0.975,0.997}
%\definecolor{shadecode}{rgb}{0.91,0.91,0.91}
%\definecolor{red_1}{rgb}{1,0.8,0.8}
%\definecolor{yellow_1}{rgb}{1,0.96,0.63}
%\definecolor{orange}{rgb}{1,0.5,0}
\definecolor{appleGray}{rgb}{0.75,0.75,0.75}
\definecolor{lightGray}{rgb}{0.975,0.975,0.975}
\definecolor{relhead}{rgb}{0.70,0.80,0.90}
\definecolor{borderGray}{rgb}{0.8,0.8,0.8}
%\definecolor{gray}{rgb}{0.975,0.975,0.975}
\definecolor{nearwhite}{rgb}{0.985,0.985,0.985}
%\definecolor{supergray}{cmyk}{0,0,0.04,0}
%\definecolor{stainlessSteel}{cmyk}{0,0,0.02,0.12}
%\definecolor{mygreen}{rgb}{0,0.6,0}
\definecolor{mygray}{rgb}{0.20,0.20,0.20}
\definecolor{mymauve}{rgb}{0.58,0,0.82}
\definecolor{excel}{rgb}{0.94, 0.94, 0.94}



\usepackage{listings}  

 \lstset{ 
  backgroundcolor=\color{white},  	 % background color; e.g., nearwhite you must add \usepackage{color} or \usepackage{xcolor}; should come as last argument
  basicstyle=\footnotesize\ttfamily,        % the size of the fonts that are used for the code
  breakatwhitespace=false,         		    % sets if automatic breaks should only happen at whitespace
  breaklines=true,                 		    % sets automatic line breaking
  framextopmargin=5pt,
  framexleftmargin=5pt, 
  framexbottommargin=5pt,
  framexrightmargin=0pt,
  framesep=0pt,
  captionpos=b,                    			% sets the caption-position to bottom
  commentstyle=\color{mygreen},    		    % comment style
  morecomment=[s]{/*}{*/},
  deletekeywords={...},            			% if want to delete keywords from the given language
  escapeinside={\%*}{*)},          			% if you want to add LaTeX within your code
  extendedchars=true,              			% lets you use non-ASCII characters; for 8-bits encodings only, does not work with UTF-8
  frame=single,	                   			% adds a frame around the code
  keepspaces=false,               			% keeps spaces in text, useful for keeping indentation of code (possibly needs columns=flexible)
  keywordstyle=\color{blue},      		% keyword style blue
  language=SQL,                 			% the language of the code
  morekeywords={ORDER, USE, DELIMITER, CALL, DECLARE, IF, ELSE, ELSEIF, WHILE, DO, LOOP, REPEAT, UNTIL, CURSOR, FOR, HANDLER, OUT, INTO, FROM, RETURNS, RETURN, FUNCTION, SHOW, TRIGGER, TRIGGERS, END$$, EVENT},   % if you want to add more keywords to the set
  numbers=none,                    			% where to put the line-numbers; possible values are (none, left, right)
  numbersep=0pt,                   			% how far the line-numbers are from the code
  numberstyle=\tiny\color{mygray}, 		    % the style that is used for the line-numbers
  rulecolor=\color{appleGray},  % appleGray     % if not set, the frame-color may be changed on line-breaks within not-black text (e.g. comments (green here))
  sensitive=true,
  showspaces=false,                			% show spaces everywhere adding particular underscores; it overrides 'showstringspaces'
  showstringspaces=false,          		    % underline spaces within strings only
  showtabs=false,                  			% show tabs within strings adding particular underscores
  stepnumber=2,                    			% the step between two line-numbers. If it's 1, each line will be numbered
  stringstyle=\color{mymauve},     		    % string literal style
  tabsize=4,	                   			% sets default tabsize to 2 spaces
  title=\lstname,                  			% show the filename of files included with \lstinputlisting; also try caption instead of title
  upquote=true,      % Straight quotes
  belowcaptionskip=0em,
  belowskip=0em
}
 



\begin{document}

%+Title
\title{Selection Operator: \texttt{WHERE} Clause}
\author{DSC 301: Lecture 5}
\date{\today}
\maketitle
%-Title

%+Abstract
%\begin{abstract}
%    There is abstract text that you should replace with your own. 
%\end{abstract}
%-Abstract

%+Contents
% \tableofcontents
%-Contents

% ---------------- %
\begin{outline}[enumerate]

\end{outline}
% ---------------- %
\begin{outline}
        
\end{outline}


% ------- Resources -------- %
% https://www.techonthenet.com/sql/not.php
% https://www.techonthenet.com/sql/like.php

% --------- Lecture Objectives ------- %
\section*{Lecture Objectives}
\begin{outline}
        \1  \texttt{WHERE} clause
        \1  SQL operators
        \1  Rename operator (alias)
\end{outline}

\hspace{-0.5cm}\rule[0.101in]{\textwidth}{0.0025in}


%\begin{definition}
%asdfasdf
%\end{definition}



\noindent Last time we covered \textit{projection} operators, $\Pi_{(a,b,c)}(R)$, that filtered attributes by selecting a subset of the columns.  In this lecture, we cover the \textit{selection} operator from \textit{relational algebra}.  Recall these operators are written as $\sigma_p(R)$ where $p$ is a predicate.  The form of this predicate is our focus.   


\hspace{-0.5cm}\rule[0.101in]{\textwidth}{0.0025in}














% -------- WHERE  -------- %
\subsection*{The \texttt{WHERE} Clause}
 
 The \verb@WHERE@ clause is used to filter records based on a conditional predicate.  A predicate is a statement containing a conditional expression that is either TRUE or FALSE. \\ %The \verb@WHERE@ clause filters the number of records returned by the query. 


\noindent The general syntax is:

% CODE: 
\begin{lstlisting}[frame=single]  
SELECT <columns> FROM <TableX> WHERE [condition];
\end{lstlisting} 

%\hspace{-0.95cm}\rule[0.101in]{\textwidth}{0.0025in}








% ---------- COMPARISONS: =, <, >, <> ---------- %
\subsubsection*{Comparison Operators}

Comparison operators include equal ($=$), non-equal ($<>$), less than ($<$), less than or equal ($<=$),  greater than ($>$), and greater than or equal ($>=$).  The equal sign is used to test equality.\\

\begin{example}
Select all carriers and flight numbers whose destination is San Francisco.    
\end{example}
\begin{lstlisting}[frame=single]  
SELECT flight FROM Flights WHERE dest = `SFO';
\end{lstlisting} 






\begin{example}
Select all flights that whose departure was delayed at least 30 minutes.
   
\end{example}

\begin{lstlisting}[frame=single]  
SELECT flight FROM Flights WHERE dep_delay >= 30;
\end{lstlisting} 


% ------------------------------------------------% 
\hspace{-0.5cm}\rule[0.101in]{\textwidth}{0.0025in}
% ------------------------------------------------% 

















% ---------- Conjunctive: AND and OR, negation  ---------- %
\subsubsection*{Conjunctive Operators}

Conjunctions are used to build more complex queries. For \textbf{example}, if we want to know the flights to San Francisco that arrive on time, we would write:



\begin{lstlisting}[frame=single]  
SELECT flight FROM Flights WHERE dest='SFO' AND arr_delay <=0;
\end{lstlisting} 



\noindent Here are a few more examples.

%%%%
\begin{example}
Find all flights going to Chicago, either O'hara or Midway.  
\end{example}


\begin{lstlisting}[frame=single]  
SELECT flight FROM Flights WHERE dest='ORD' OR dest='MDW';
\end{lstlisting} 






%%%%
\begin{example}
Find all flight IDs going to Chicago Midway on Christmas day.    
\end{example}


\begin{lstlisting}[frame=single]  
SELECT fid FROM Flights WHERE dest='MDW' AND month=12 AND day=25;
\end{lstlisting} 




















% ---------- LOGICAL  ---------- %
\subsubsection*{Logical Operators}

Logical operators used with the \verb@WHERE@ clause include\footnote{ALL, ANY, EXIST, SOME, UNIQUE will be used with subqueries.    }: 
\begin{table}[h!]
    \centering
    \begin{tabular}{lll}
    \hline
        \texttt{IS NULL} \hspace{0.5cm}  & \texttt{BETWEEN}\hspace{0.5cm}  & \texttt{IN}  \\
         \texttt{LIKE} \hspace{0.5cm}  & \texttt{EXISTS}  \hspace{0.5cm}  &  \texttt{UNIQUE} \\
          \texttt{ALL}  \hspace{0.5cm}  & \texttt{SOME}  \hspace{0.5cm}  & \texttt{ANY}  \\
    \hline 
    \end{tabular}
    \caption{Logical Operators in SQL}
    \label{tab:logical}
\end{table}


% See: https://www.dummies.com/programming/sql/how-to-use-exists-unique-distinct-and-overlaps-in-sql-statements/







\begin{example}
Select fights with delays ranging between 10 and 20 minutes.
\end{example}

\begin{lstlisting}[frame=single]  
SELECT 
    flight, dep_delay
FROM
    Flights
WHERE
    dep_delay BETWEEN 10 AND 30;
\end{lstlisting} 




% --------- IN
\subsubsection*{IN}
Records where field is in a list of criteria.  For example, flights whose destination matches one of many.  For  \textbf{example}, flights with destination San Francisco, Los Angles, or Oakland California.   



\begin{lstlisting}[frame=single]  
SELECT fid FROM Flights WHERE dest IN ('SFO','LAX','OAK');
\end{lstlisting} 



%--------  LIKE
\subsubsection*{LIKE}
The \texttt{LIKE} condition is used in the \texttt{WHERE} clause and allows the use of wildcards to perform pattern matching in a query.  Wildcards include: 


%    %  Allows you to match any string of any length (including zero length)
%    _ Allows you to match on a single character
 
 \begin{table}
  \centering
    \begin{tabular}{cll}
     \hline
   \textbf{\texttt{LIKE }Wildcards} & & \textbf{Description} \\ 
    \hline
        \texttt{\%} & &  Match any string of any length (including zero length)  \\
         \texttt{\_} & & Match on a single character \\

    \hline 
    \end{tabular}
    \caption{Wildcards for \texttt{LIKE}}
 \end{table} 






\begin{lstlisting}[frame=single]  
SELECT dest FROM Flights where dest like 'H%';
\end{lstlisting} 


\begin{lstlisting}[frame=single]  
SELECT dest FROM Flights where dest like '_N_';
\end{lstlisting} 



% ------------------------------------------------% 
\hspace{-0.5cm}\rule[0.101in]{\textwidth}{0.0025in}
% ------------------------------------------------% 





















\subsubsection*{Negation}
How about not?  Many times you want the compliment of a set or results that
do \textbf{not} include some parameter.  Equality is negated using \texttt{<>}
(or \texttt{!=}).   See the next example.





%%%%
\begin{example}
Find all flight IDs not going to Chicago Midway on Christmas day.    
\end{example}


\begin{lstlisting}[frame=single]  
SELECT fid FROM Flights WHERE dest<>'MDW' AND month=12 AND day=25;
\end{lstlisting} 



\noindent Negate logical operators using the keyword \texttt{NOT}.  It can be used with any logical operator, e.g., \texttt{NOT LIKE, NOT IN, NOT BETWEEN, NOT\ EXISTS, NOT\ ANY}, etc.  Here are two examples using \texttt{NOT}\footnote{\texttt{NOT} is often used together with the \texttt{IS NULL} operator.}. 

%%%%
\begin{example}
Find all carriers departing SEA excluding AS, DL, WN, and UA.   
\end{example}


\begin{lstlisting}[frame=single]  
SELECT 
    carrier
FROM
    Flights
WHERE
    carrier NOT IN ('AS' , 'DL', 'WN', 'UA'); 
\end{lstlisting} 







%%%%
\begin{example}
Find flgiths departing SEA whose distances are less than 500 or greater than 1000 using \texttt{NOT BETWEEN}.  Can you do this another way?     
\end{example}


\begin{lstlisting}[frame=single]  
SELECT 
    flight
FROM
    Flights
WHERE
    distance NOT BETWEEN 500 AND 1000; 
\end{lstlisting} 




\begin{lstlisting}[frame=single]  
SELECT flight FROM Flights WHERE distance<500 OR distance > 1000; 
\end{lstlisting} 









% ------------------------------------------------% 
\hspace{-0.5cm}\rule[0.101in]{\textwidth}{0.0025in}
% ------------------------------------------------% 











% ---------- Arithmetic  ---------- %
% \subsubsection*{Arithmetic Operators}


% Do with Aggregates

% ------------------------------------------------% 
% \hspace{-0.5cm}\rule[0.101in]{\textwidth}{0.0025in}
% ------------------------------------------------% 

















% ---------- Rename  ---------- %
\subsection*{Rename Operator}

We studied \textit{rename} operators, $\rho_{a}(R)$, in the section on relational algebra. The idea is simple, sometimes we want to rename an attribute or relation\footnote{We will not rename tables until working with multiple tables.}.  Here is an example illustrating the rename operator.  This operation is also called \textbf{aliasing}. 





%%%%
\begin{example}[Column alias]
List the departure delays from flights to San Francisco in June.  Rename departure delay field to just ``Delay''.  
\end{example}


\begin{lstlisting}[frame=single]  
SELECT 
    dep_delay AS Delay
FROM
    Flights
WHERE
    dest = 'SFO' AND month = 6; 
\end{lstlisting} 

\noindent \textbf{Note 1}: This displays all delays (even if early).  How can you change it to display only flights that were actually late departing?\\

 
 
\noindent \textbf{Note 2}: If you want to rename column with multiple words (e.g., Departure Delay), you must use single quotations. The example above would be:  
 

\begin{lstlisting}[frame=single]
SELECT 
    dep_delay AS 'Departure Delay'
FROM
    Flights
WHERE
    dest = 'SFO' AND month = 6; 

\end{lstlisting} 



% ------------------------------------------------% 
\hspace{-0.5cm}\rule[0.101in]{\textwidth}{0.0025in}
% ------------------------------------------------% 














  
%+Bibliography
%\begin{thebibliography}{99}
%\bibitem{Label1} ...
%\bibitem{Label2} ...
%\end{thebibliography}
%-Bibliography



 

\end{document}

