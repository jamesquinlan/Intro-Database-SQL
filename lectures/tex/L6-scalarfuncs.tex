%&latex
\documentclass{article}

 
\usepackage{amsthm}
\newtheorem{definition}{Definition}
\newtheorem{example}{Example}


\usepackage{outlines}
\usepackage{enumitem}
\setenumerate[1]{label=\arabic*.}
\setenumerate[2]{label=\alph*.}
\setenumerate[3]{label=\roman*.}
\setenumerate[4]{label=\alph*.}

\usepackage{graphicx}
\DeclareGraphicsRule{.tif}{png}{.png}{`convert #1 `dirname #1`/`basename #1 .tif`.png}



\usepackage{framed}  % https://latexcolor.com
\usepackage{xcolor}


\usepackage{tcolorbox, graphicx}
\usepackage{colortbl}
\usepackage{textcomp}

% Colors
%\definecolor{anti-flashwhite}{rgb}{0.95, 0.95, 0.96}
%\definecolor{magnolia}{rgb}{0.97, 0.96, 1.0}
\definecolor{antiquewhite}{rgb}{0.96, 0.96, 0.96} % {0.98, 0.92, 0.84}
%\definecolor{shadecolor}{rgb}{0.95,0.975,0.997}
%\definecolor{shadecode}{rgb}{0.91,0.91,0.91}
%\definecolor{red_1}{rgb}{1,0.8,0.8}
%\definecolor{yellow_1}{rgb}{1,0.96,0.63}
%\definecolor{orange}{rgb}{1,0.5,0}
\definecolor{appleGray}{rgb}{0.75,0.75,0.75}
\definecolor{lightGray}{rgb}{0.975,0.975,0.975}
\definecolor{relhead}{rgb}{0.70,0.80,0.90}
\definecolor{borderGray}{rgb}{0.8,0.8,0.8}
%\definecolor{gray}{rgb}{0.975,0.975,0.975}
\definecolor{nearwhite}{rgb}{0.985,0.985,0.985}
%\definecolor{supergray}{cmyk}{0,0,0.04,0}
%\definecolor{stainlessSteel}{cmyk}{0,0,0.02,0.12}
%\definecolor{mygreen}{rgb}{0,0.6,0}
\definecolor{mygray}{rgb}{0.20,0.20,0.20}
\definecolor{mymauve}{rgb}{0.58,0,0.82}
\definecolor{excel}{rgb}{0.94, 0.94, 0.94}



\usepackage{listings}  

 \lstset{ 
  backgroundcolor=\color{white},  	 % background color; e.g., nearwhite you must add \usepackage{color} or \usepackage{xcolor}; should come as last argument
  basicstyle=\footnotesize\ttfamily,        % the size of the fonts that are used for the code
  breakatwhitespace=false,         		    % sets if automatic breaks should only happen at whitespace
  breaklines=true,                 		    % sets automatic line breaking
  framextopmargin=5pt,
  framexleftmargin=5pt, 
  framexbottommargin=5pt,
  framexrightmargin=0pt,
  framesep=0pt,
  captionpos=b,                    			% sets the caption-position to bottom
  commentstyle=\color{mygreen},    		    % comment style
  morecomment=[s]{/*}{*/},
  deletekeywords={...},            			% if want to delete keywords from the given language
  escapeinside={\%*}{*)},          			% if you want to add LaTeX within your code
  extendedchars=true,              			% lets you use non-ASCII characters; for 8-bits encodings only, does not work with UTF-8
  frame=single,	                   			% adds a frame around the code
  keepspaces=false,               			% keeps spaces in text, useful for keeping indentation of code (possibly needs columns=flexible)
  keywordstyle=\color{blue},      		% keyword style blue
  language=SQL,                 			% the language of the code
  morekeywords={ORDER, USE, DELIMITER, CALL, DECLARE, IF, ELSE, ELSEIF, WHILE, DO, LOOP, REPEAT, UNTIL, CURSOR, FOR, HANDLER, OUT, INTO, FROM, RETURNS, RETURN, FUNCTION, SHOW, TRIGGER, TRIGGERS, END$$, EVENT},   % if you want to add more keywords to the set
  numbers=none,                    			% where to put the line-numbers; possible values are (none, left, right)
  numbersep=0pt,                   			% how far the line-numbers are from the code
  numberstyle=\tiny\color{mygray}, 		    % the style that is used for the line-numbers
  rulecolor=\color{appleGray},  % appleGray     % if not set, the frame-color may be changed on line-breaks within not-black text (e.g. comments (green here))
  sensitive=true,
  showspaces=false,                			% show spaces everywhere adding particular underscores; it overrides 'showstringspaces'
  showstringspaces=false,          		    % underline spaces within strings only
  showtabs=false,                  			% show tabs within strings adding particular underscores
  stepnumber=2,                    			% the step between two line-numbers. If it's 1, each line will be numbered
  stringstyle=\color{mymauve},     		    % string literal style
  tabsize=4,	                   			% sets default tabsize to 2 spaces
  title=\lstname,                  			% show the filename of files included with \lstinputlisting; also try caption instead of title
  upquote=true,      % Straight quotes
  belowcaptionskip=0em,
  belowskip=0em
}
 




\begin{document}

%+Title
\title{String and Scalar Functions}
\author{DSC 301: Lecture 6}
\date{February 10, 2021} % \today
\maketitle
%-Title

%+Abstract
%\begin{abstract}
%    There is abstract text that you should replace with your own. 
%\end{abstract}
%-Abstract



%+Contents
% \tableofcontents
%-Contents

% ---------------- %
\begin{outline}[enumerate]

\end{outline}
% ---------------- %
\begin{outline}
        
\end{outline}




% --------- Lecture Objectives ------- %
\section*{Lecture Objectives}
\begin{outline}
        \1 String functions
        \1 Arithmetic functions
        % \1 Date functions
\end{outline}

% --------------------------------------------------
\hspace{-0.5cm}\rule[-0.101in]{\textwidth}{0.0025in}
% --------------------------------------------------
% 
% 
%\begin{definition}
%asdfasdf
%\end{definition}









% --------   -------- %
\section*{String Functions}
 
\textbf{String functions} include such functions concatenation, trimming, locating substrings, obtaining string length, replacing parts, capitalizing, padding, etc.  



\subsection*{Concatenation}


\begin{outline}
    \1 \texttt{CONCAT(str1,str2)}: Concatenates two or more strings.  If one of the strings is \texttt{null}, the entire result is \texttt{null}.


\begin{lstlisting}[frame=single]  
SELECT CONCAT(carrier,flight) as Flight from Flights;
\end{lstlisting} 
Note the use of aliasing in the example above.  



    \1 \texttt{CONCAT\_WS(sep,str1,str2)}: Concatenates two or more strings with a separator (\texttt{sep}).  If \texttt{sep} is null, the result is \texttt{null}.  The example below will place a hyphen between the carrier and flight number.  


\begin{lstlisting}[frame=single]  
SELECT CONCAT_WS('-',carrier,flight) as Flight from Flights;
\end{lstlisting}  

\end{outline}

 
% --------------------------------------------------
\hspace{-0.5cm}\rule[-0.101in]{\textwidth}{0.0025in}
% --------------------------------------------------
  
  
  
  
  
% LTRIM
\subsection*{Trimming}

\begin{outline}
 
        % LTRIM
    \1 \texttt{LTRIM(str)}: Returns the string (\texttt{str}) with left spaces removed.  That is trims left spacing.  
   
   
\begin{lstlisting}[frame=single]  
SELECT LTRIM('  American Airlines') as Airline;
\end{lstlisting} 




        % RTRIM
    \1 \texttt{RTRIM(str)}: Returns the string (\texttt{str}) with right spaces removed.  That is trims right spacing.    
   
   
\begin{lstlisting}[frame=single]  
SELECT RTRIM('  American Airlines  ') as Airline;
\end{lstlisting} 
Note left spaces still exist in the result of this example.  





        % TRIM
    \1 \texttt{TRIM([both|leading|trailing] [remove FROM] str)}: Returns
the string (\texttt{str}) with \texttt{remove} eliminated from the   begining
or ending (or both).  If \texttt{remove} is omitted, only spaces are removed.
   
   
\begin{lstlisting}[frame=single]  
SELECT TRIM(LEADING '---' FROM '---American Airlines') as Airline;
\end{lstlisting} 



\end{outline}
 
  
  

% --------------------------------------------------
\hspace{-0.5cm}\rule[-0.101in]{\textwidth}{0.0025in}
% --------------------------------------------------
    
  
  
  
  
  
  
  
  
  
  
  
  
  

\subsection*{Length of strings}

\begin{outline}
   \1 \texttt{LENGTH(str)}: returns number of characters in the string.
   
   
\begin{lstlisting}[frame=single]  
SELECT LENGTH('American Airlines') as 'Length of Name';
\end{lstlisting} 

Note difference in length between the previous query and the following.   
   
\begin{lstlisting}[frame=single]  
SELECT LENGTH('   American Airlines') as 'Length of Name';
\end{lstlisting}

\end{outline}
 
  
  

% --------------------------------------------------
\hspace{-0.5cm}\rule[-0.101in]{\textwidth}{0.0025in}
% --------------------------------------------------
    
  
  
  
  
  
  
  






  

\subsection*{Substrings}

\begin{outline}

 % --------- LEFT ---------- %  
   \1 \texttt{LEFT(str,length)}: Returns a specified number of characters from the beginning of the  string.
        
         
\begin{lstlisting}[frame=single]  
SELECT LEFT('American Airlines',4) as 'Begining 4 characters';
\end{lstlisting}

        
\begin{lstlisting}[frame=single]  
SELECT LEFT(tailnum,2) as 'Part tailnum' from Flights;
\end{lstlisting}

         
         
         
         
         
    % --------- RIGHT ---------- %     
   \1 \texttt{RIGHT(str,length)}: Returns a specified number of characters
from the end of the  string.

       
\begin{lstlisting}[frame=single]  
SELECT RIGHT('American Airlines',5) as 'Last 5 characters';
\end{lstlisting}





      
    % --------- SUBSTRING ---------- %     
   \1 \texttt{SUBSTRING(str,start,[length])}: Returns the a specified number of characters starting from the \texttt{start} position.  If \texttt{length} is empty, it returns from the starting position to the end.    
       
\begin{lstlisting}[frame=single]  
SELECT SUBSTRING('American Airlines', 4,5) as 'Substring';
\end{lstlisting}


       
\begin{lstlisting}[frame=single]  
SELECT SUBSTRING(tailnum, 2,3) as 'PartTail' from Flights;
\end{lstlisting}

       

  
  
  
  
  
  
  
      
    % --------- SUBSTRING INDEX ---------- %     
   \1 \texttt{SUBSTRING\_INDEX(str,delimiter,number)}: Returns  the substring before the specified \texttt{number} of occurences of the delimiter.  When \texttt{number} is positive, it starts from the beginning and when negative from the end. Used to parse strings. 
       
\begin{lstlisting}[frame=single]  
SELECT SUBSTRING_INDEX('https://www.une.edu', '.',-2) as 'Domain Name';
\end{lstlisting}


 
   


      
    % --------- LOCATE ---------- %     
   \1 \texttt{LOCATE(lookfor,search,[start])}: Returns the \textbf{position} of the first occurrence of \texttt{lookfor} in the \texttt{search} string starting at the optional parameter \texttt{start}.  If \texttt{start} is empty, the search starts at the beginning of the string.  
       
\begin{lstlisting}[frame=single]  
SELECT LOCATE('irl','American Airlines') as 'Position of irl';
\end{lstlisting}

   
      
\begin{lstlisting}[frame=single]  
 SELECT LOCATE('O',dest,2) as 'Position of O' from Flights;
\end{lstlisting}

   
   
   
   
   
   
   
      
    % --------- REPLACE ---------- %     
   \1 \texttt{REPLACE(search,find, replace)}: Returns the \texttt{search} string with the characters \texttt{find} replaced by \texttt{replace}.  
       
\begin{lstlisting}[frame=single]  
SELECT REPLACE('American Airlines','Air','Van') as 'New Str';
\end{lstlisting}

    
    
    

   
      
    % --------- INSERT ---------- %     
   \1 \texttt{INSERT(str,start, length,insert)}: Returns the \texttt{search}
string with the characters \texttt{find} replaced by \texttt{replace}.  
       
\begin{lstlisting}[frame=single]  
SELECT INSERT('American Airlines',5,3,'Van') as 'New Str';
\end{lstlisting}

    
    
    
    
   
      
    % --------- UPPER ---------- %     
   \1 \texttt{UPPER(str)}: Returns the \texttt{str}
converted to uppercase letters.  
       
\begin{lstlisting}[frame=single]  
SELECT UPPER('American Airlines') as 'All UPPERCASE';
\end{lstlisting}

    
    
    
    
   
      
    % --------- LOWER ---------- %     
   \1 \texttt{LOWER(str)}:  Returns the \texttt{str}
converted to lowercase letters.
       
\begin{lstlisting}[frame=single]  
SELECT LOWER('American Airlines') as 'All lowercase';
\end{lstlisting}

    
    
    
    
    
    
   
      
    % --------- LPAD ---------- %     
   \1 \texttt{LPAD(str, length, char)}: Returns a string of length \texttt{length} padded on the \textbf{left} with \texttt{char}.  If \texttt{str} is longer than the \texttt{length}, the \texttt{str} is truncated and not padding occurs.      
       
\begin{lstlisting}[frame=single]  
SELECT LPAD('American Airlines',20,'.') as 'Padded String'; 
\end{lstlisting}



   
\begin{lstlisting}[frame=single]  
SELECT LPAD('American Airlines',15,'.') as 'Truncated String'; 
\end{lstlisting}

        
    
    
    
   
      
    % --------- RPAD ---------- %     
  \1 \texttt{RPAD(str, length, char)}: Returns a string of length \texttt{length}
padded on the \textbf{right} with \texttt{char}.  If \texttt{str} is longer than the \texttt{length}, the \texttt{str} is truncated and not padding occurs.  
   
       
\begin{lstlisting}[frame=single]  
SELECT RPAD('American Airlines',20,'.') as 'Padded String'; 
\end{lstlisting}

    
    
    
    
    
    
   
      
    % --------- SPACE ---------- %     
   \1 \texttt{SPACE(number)}: Returns the specified \texttt{number} of space characters.
   
       
\begin{lstlisting}[frame=single]  
SELECT CONCAT('American',  SPACE(10), 'Airlines') as 'New Str';
\end{lstlisting}

    
    
    
    
   
      
    % --------- REPLACE ---------- %     
   \1 \texttt{REPLACE(str,replace,with)}: Returns the string (\texttt{str}) with all occurances of the substring \texttt{replace}, replaced by \texttt{with}.  For example, the code below will change `American Airlines' to `American Vanlines'.   
       
\begin{lstlisting}[frame=single]  
SELECT REPLACE('American Airlines','Air','Van') as 'New Str';
\end{lstlisting}

    
    
   
      
    % --------- REVERSE ---------- %     
   \1 \texttt{REVERSE(str)}: Returns  \texttt{str} with characters in reverse order. 

       
\begin{lstlisting}[frame=single]  
SELECT REVERSE('American Airlines') as 'Reverse Str';
\end{lstlisting}

    
    
    
    
    
   
      
    % --------- REPEAT ---------- %     
   \1 \texttt{REPEAT(str,times)}: Returns the \texttt{str}
repeated  \texttt{times} number of times.  
       
\begin{lstlisting}[frame=single]  
SELECT REPEAT('Air',3) as 'Repeat Str 3 times';
\end{lstlisting}


\end{outline}
  

% --------------------------------------------------
\hspace{-0.5cm}\rule[-0.101in]{\textwidth}{0.0025in}
% --------------------------------------------------
    
  






  

\section*{Numeric Functions}

\subsection*{Basic Math}
Basic math function include addition, subtraction, multiplication,  division, modulo, and exponentiation.

 
\begin{outline}

    \1 \texttt{+}: Adds values.  For example to add 30 minutes to the air time of a flight, we write:

\begin{lstlisting}[frame=single]  
SELECT air_time + 30 as 'New Air Time' FROM Flights;
\end{lstlisting}


    \1 \texttt{-}: Subtraction.  Similar to subtract 30 minutes.  

\begin{lstlisting}[frame=single]  
SELECT air_time - 30 as 'New Air Time' FROM Flights;
\end{lstlisting}


    \1 \texttt{*}: Multiply.  Scales values.  For example, to convert to flight distance in miles to kilometers.  

\begin{lstlisting}[frame=single]  
SELECT 1.609344*distance as 'Distance (km)' FROM Flights;
\end{lstlisting}


    \1 \texttt{/}: Divide.  Similar usage to multiplication.  For example, lets determine the flight time in hours.  


\begin{lstlisting}[frame=single]  
SELECT air_time / 60 as 'Travel time (hours)' FROM Flights;
\end{lstlisting}

    
    
    %IF distance was stored in kilometers in the table, then to convert to miles we would divide.

%\begin{lstlisting}[frame=single]  
%SELECT distance/1.609344 as 'Distance (miles)' FROM Flights;
%\end{lstlisting}
   
\end{outline}
 


% --------------------------------------------------
\hspace{-0.5cm}\rule[-0.101in]{\textwidth}{0.0025in}
% --------------------------------------------------
    
  
  
   
 
 
 
  
 \subsection*{Scalar Functions}
 
\begin{outline}
   \1 \texttt{abs(x)}: absolute value, e.g., abs(-17.4)
   \1 \texttt{cbrt(x)}: cube root, e.g., cbrt(8)
   \1 \texttt{ceil(x)}: smallest integer not less than argument, e.g., ceil(-42.8)
   \1 \texttt{ceiling(x)}:same as \texttt{ceil(x)}
   \1 \texttt{degree(x)}: radians to degrees
   \1 \texttt{floor(x)}: largest integer not greater than argument, e.g.,  floor(-42.8)
   \1 \texttt{exp(x)}: exp(1) is the natural number $e$
   \1 \texttt{ln(x)}: natural logarithm, e.g., ln(2.0)
   \1 \texttt{log(x)}: base 10 logarithm,  e.g., log(100)
   \1 \texttt{mod(y,x)}: remainder of y/x, e.g., mod(9,4)
   \1 \texttt{pi()}: $\pi$
   \1 \texttt{pow(x,y)}: raise a number to exponent y, e.g., pow(9.0, 3.0)
   \1 \texttt{power(x,y)}: same as \texttt{pow}
   \1 \texttt{radians(x)}: degree to radians
   \1 \texttt{rand()}: random value between 0.0 and 1.0
   \1 \texttt{round(x,p)}: round to p decimal places, e.g., round(42.4382, 2)
   \1 \texttt{sign(x)}: sign of the argument, i.e., $(-1, 0, +1)$
   \1 \texttt{sqrt(x)}: square root, e.g., sqrt(2.0)
   \1 \texttt{truncate(x,p)}: truncate to s decimal places, trunc(42.4382,2)
\end{outline}
 
  
  
   

% --------------------------------------------------
\hspace{-0.5cm}\rule[-0.101in]{\textwidth}{0.0025in}
% --------------------------------------------------
    
  
 
 
  
 \subsection*{Date Functions}
 
 Not covered in this lecture.     
 
 
 
 
 
 
 
 
 
  
 \subsection*{Next time}
 
 Aggregate functions  
 
 
 
 
 
 
 
 
 
 
 
 
 
 
  
   
  
  
  
  
  
  
  
  
  
  
  
  
  
  
  
  
  
%+Bibliography
%\begin{thebibliography}{99}
%\bibitem{Label1} ...
%\bibitem{Label2} ...
%\end{thebibliography}
%-Bibliography



 

\end{document}
