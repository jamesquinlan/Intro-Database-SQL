% &latex

% Wed Apr  09 10:00:45 EDT 2021
%+++++++++++++++++++++++++++++++++++++++++++++++++++++++++++++++++++++
% SUMMARY   :  Introduction to Database
%           :  Views - Stored Procedures
%	    :
%           :  Quinlan, J 
%           :  University of New England
%+++++++++++++++++++++++++++++++++++++++++++++++++++++++++++++++++++++

\documentclass{article}

 
\usepackage{amsthm}
\newtheorem{definition}{Definition}
\newtheorem{example}{Example}


% OUTLINE
\usepackage{outlines}
\usepackage{enumitem}
\setenumerate[1]{label=\arabic*.}
\setenumerate[2]{label=\alph*.}
\setenumerate[3]{label=\roman*.}
\setenumerate[4]{label=\alph*.}

\usepackage{graphicx}
\DeclareGraphicsRule{.tif}{png}{.png}{`convert #1 `dirname #1`/`basename #1 .tif`.png}


% CODE (SQL)

\usepackage{framed}  % https://latexcolor.com
\usepackage{xcolor}


\usepackage{tcolorbox, graphicx}
\usepackage{colortbl}
\usepackage{textcomp}

% Colors
%\definecolor{anti-flashwhite}{rgb}{0.95, 0.95, 0.96}
%\definecolor{magnolia}{rgb}{0.97, 0.96, 1.0}
\definecolor{antiquewhite}{rgb}{0.96, 0.96, 0.96} % {0.98, 0.92, 0.84}
%\definecolor{shadecolor}{rgb}{0.95,0.975,0.997}
%\definecolor{shadecode}{rgb}{0.91,0.91,0.91}
%\definecolor{red_1}{rgb}{1,0.8,0.8}
%\definecolor{yellow_1}{rgb}{1,0.96,0.63}
%\definecolor{orange}{rgb}{1,0.5,0}
\definecolor{appleGray}{rgb}{0.75,0.75,0.75}
\definecolor{lightGray}{rgb}{0.975,0.975,0.975}
\definecolor{relhead}{rgb}{0.70,0.80,0.90}
\definecolor{borderGray}{rgb}{0.8,0.8,0.8}
%\definecolor{gray}{rgb}{0.975,0.975,0.975}
\definecolor{nearwhite}{rgb}{0.985,0.985,0.985}
%\definecolor{supergray}{cmyk}{0,0,0.04,0}
%\definecolor{stainlessSteel}{cmyk}{0,0,0.02,0.12}
%\definecolor{mygreen}{rgb}{0,0.6,0}
\definecolor{mygray}{rgb}{0.20,0.20,0.20}
\definecolor{mymauve}{rgb}{0.58,0,0.82}
\definecolor{excel}{rgb}{0.94, 0.94, 0.94}



\usepackage{listings}  

 \lstset{ 
  backgroundcolor=\color{white},  	 % background color; e.g., nearwhite you must add \usepackage{color} or \usepackage{xcolor}; should come as last argument
  basicstyle=\footnotesize\ttfamily,        % the size of the fonts that are used for the code
  breakatwhitespace=false,         		    % sets if automatic breaks should only happen at whitespace
  breaklines=true,                 		    % sets automatic line breaking
  framextopmargin=5pt,
  framexleftmargin=5pt, 
  framexbottommargin=5pt,
  framexrightmargin=0pt,
  framesep=0pt,
  captionpos=b,                    			% sets the caption-position to bottom
  commentstyle=\color{mygreen},    		    % comment style
  morecomment=[s]{/*}{*/},
  deletekeywords={...},            			% if want to delete keywords from the given language
  escapeinside={\%*}{*)},          			% if you want to add LaTeX within your code
  extendedchars=true,              			% lets you use non-ASCII characters; for 8-bits encodings only, does not work with UTF-8
  frame=single,	                   			% adds a frame around the code
  keepspaces=false,               			% keeps spaces in text, useful for keeping indentation of code (possibly needs columns=flexible)
  keywordstyle=\color{blue},      		% keyword style blue
  language=SQL,                 			% the language of the code
  morekeywords={ORDER, USE, DELIMITER, CALL, DECLARE, IF, ELSE, ELSEIF, WHILE, DO, LOOP, REPEAT, UNTIL, CURSOR, FOR, HANDLER, OUT, INTO, FROM, RETURNS, RETURN, FUNCTION, SHOW, TRIGGER, TRIGGERS, END$$, EVENT},   % if you want to add more keywords to the set
  numbers=none,                    			% where to put the line-numbers; possible values are (none, left, right)
  numbersep=0pt,                   			% how far the line-numbers are from the code
  numberstyle=\tiny\color{mygray}, 		    % the style that is used for the line-numbers
  rulecolor=\color{appleGray},  % appleGray     % if not set, the frame-color may be changed on line-breaks within not-black text (e.g. comments (green here))
  sensitive=true,
  showspaces=false,                			% show spaces everywhere adding particular underscores; it overrides 'showstringspaces'
  showstringspaces=false,          		    % underline spaces within strings only
  showtabs=false,                  			% show tabs within strings adding particular underscores
  stepnumber=2,                    			% the step between two line-numbers. If it's 1, each line will be numbered
  stringstyle=\color{mymauve},     		    % string literal style
  tabsize=4,	                   			% sets default tabsize to 2 spaces
  title=\lstname,                  			% show the filename of files included with \lstinputlisting; also try caption instead of title
  upquote=true,      % Straight quotes
  belowcaptionskip=0em,
  belowskip=0em
}
 




\begin{document}

%+Title
\title{Stored Procedures}
\author{DSC 301: Lecture 16}
\date{April 9, 2021} % \today
\maketitle
%-Title

%+Abstract
\begin{abstract}
\noindent  \textbf{Stored programs} are extensions to standard SQL, specifically they include:  \textit{stored procedures, stored functions, triggers}, and \textit{events}. Stored programs can control the flow of execution, can be called from other applications or within a SQL statement.  \textit{Triggers} are executed automatically when an \texttt{INSERT}, \texttt{UPDATE}, or \texttt{DELETE} statement is run on a table.  \textit{Events} are scheduled and execute at that time.  
\end{abstract}
%-Abstract


%+Contents
% \tableofcontents
%-Contents


% --------- OBJECTIVES ------- %
\section*{Lecture Objectives}
\noindent In this lesson you will learn about \textit{stored procedures}.  Specifically, we cover:  %  what they are, how to use them, and when to use them.  

\begin{outline}
	\1 Why use stored procedures?  % 1. faster, 2. help debug select statements by returning messages
        \1  Syntax to create and call a stored procedure
     		\2 Flow control keywords
		\2 Use \texttt{INTO} clause to store results in variables
		
	 
\end{outline}

% --------------------------------------------------
\hspace{-0.5cm}\rule[-0.101in]{\textwidth}{0.0025in}
% --------------------------------------------------
% 







% --------- SECTION: Why use stored procedures? ------- %
\subsection*{Why use stored procedures?}
There are four basic reasons to use stored procedures: 
\begin{outline}
        \1  Simplicity - complicated workflows with multiple queries can be bundled into one procedure.  
        		\2 For example, processing a customer order involves computing tax, total, and shipping as well as updating inventory.
	\1 Consistency - Writing a query one time vs. 100 times  reduces the probability of errors.  All users execute the same procedure (SQL statements), the results will be the same each time.  
	
	\1 Security - stored procedures can be executed without user having access to underlying data to prevent data corruption (either by accident or intention)

	\1 Increase performance - compiled and stored 
     		 	 
\end{outline}


\noindent NOTE: There are a few drawbacks to stored procedures.  First,  different DBMS use different syntax making portability difficult.  Second, stored procedures add a level of complexity that require greater degree of skill and knowledge than simple SQL statements.  

% --------------------------------------------------
\hspace{-0.5cm}\rule[-0.101in]{\textwidth}{0.0025in}
% --------------------------------------------------
  
  












% --------- SECTION: Syntax to create and call a stored procedure ------- %
\subsection*{Syntax to create and call a stored procedure}

Syntax \textbf{template} for creating a stored procedure without input parameters.

\begin{lstlisting}[frame=single]  
USE whichDatabase;

# Change the delimiter because semi-colon used inside
DELIMITER $$

CREATE PROCEDURE name_of_proc()

BEGIN
	# Declare local variables if needed
	DECLARE  localVar Type;
	
	# Write SQL statement(s)
	
END$$

# Change delimiter back to semicolon
DELIMITER ;
\end{lstlisting} 

 
 % ------------------------- CALL -------------------------  %
\subsubsection*{How to call a stored procedure}

Call the stored procedure using the \texttt{CALL} keyword.  
\begin{lstlisting}[frame=single]  
CALL name_of_proc();
\end{lstlisting} 



 % ------------------------- Example -------------------------  %
\begin{example}
Create a stored procedure that displays a message (i.e., no database).
\end{example}
 
 \begin{lstlisting}[frame=single]  
DELIMITER $$

CREATE PROCEDURE msg()

BEGIN
	SELECT 'This is a message that will be displayed' as Message;
END$$

DELIMITER ;
\end{lstlisting} 
 
 
 
 
 
 \begin{example}
 Create and call a stored procedure \texttt{test()} that prints the message, ``This is a stored procedure."
 \end{example}
Using our template

\begin{lstlisting}[frame=single]  
USE dbsoln_Store;

# Change the delimiter because semi-colon used inside
DELIMITER $$

CREATE PROCEDURE test()

BEGIN
	SELECT "This is a stored procedure."	
END$$

# Change delimiter back to semicolon
DELIMITER ;

\end{lstlisting} 

Then, \texttt{CALL test()};





% --------------------------------------------------
\hspace{-0.5cm}\rule[-0.101in]{\textwidth}{0.0025in}
% --------------------------------------------------
  
  
\subsubsection*{ \textbf{Template} for creating a stored procedure WITH input parameters}

 

\begin{lstlisting}[frame=single]  
USE whichDatabase;

# Change the delimiter because semi-colon used inside
DELIMITER $$

CREATE PROCEDURE name_of_proc(
	IN 	x 			type,
	IN 	y			type
)

BEGIN
	# Declare local variables if needed
	DECLARE  localVar Type;
	
	# Write SQL statement(s)
	
END$$

# Change delimiter back to semicolon
DELIMITER ;
\end{lstlisting} 





   
  
  
  
  
  
  
  
  
  
  
  
  
  
  
  
  
  
  
  
  
% --------------------------------------------------
\hspace{-0.5cm}\rule[-0.101in]{\textwidth}{0.0025in}
% --------------------------------------------------
  
  
\subsubsection*{ \textbf{Template} WITH input and OUTPUT parameters}

 

\begin{lstlisting}[frame=single]  
USE whichDatabase;

# Change the delimiter because semi-colon used inside
DELIMITER $$

CREATE PROCEDURE name_of_proc(
	IN 		x 			type,
	IN 		y			type, 
	OUT 	z			type
)

BEGIN
	# Declare local variables if needed
	DECLARE  localVar Type;
	
	# Write SQL statement(s)
	
END$$

# Change delimiter back to semicolon
DELIMITER ;
\end{lstlisting} 






\subsection*{Declare and set variables}

A \textit{variable} stores a value that can be used and changed during execution of a stored procedure.  First, a (local) variable must be declared before it can be used.  To declare a variable, use the \texttt{DECLARE} keyword after the \texttt{BEGIN} statement of the procedure body.  A data type that matches the column type must be specified for each declared variable.  For example, a variable for average price should be \texttt{DECIMAL(9,2)}.  Data types can be any data type used in column definition (e.g., \texttt{INT},  \texttt{VARCHAR()},  \texttt{DECIMAL()},  etc.).  A default value can be assigned when declaring variables by using the \texttt{DEFAULT} keyword.  For example, \texttt{DECLARE minMSRP DECIMAL(9,2) DEFAULT 1.00}, could be used to declare a minimum MSRP for the variable \texttt{minMSRP} to one dollar.



 % ------------------------- Example -------------------------  %
\begin{example}
Create a stored procedure that displays a message (i.e., no database).
\end{example}
 
 \begin{lstlisting}[frame=single]  
DELIMITER $$

CREATE PROCEDURE msg()

BEGIN
	SELECT 'This is a message that will be displayed' as Message;
END$$

DELIMITER ;
\end{lstlisting} 





   
  
  
  % DELCARE x, y, z DECIMAL(5,2)
  
  % select a, b, c INTO  x, y, z FROM Table WHERE condition.
  
  
  
  
  
% --------------------------------------------------
\hspace{-0.5cm}\rule[-0.101in]{\textwidth}{0.0025in}
% --------------------------------------------------
  
  
\section*{Flow Control}

Flow control statements include branching and looping statements.  The following 
  
 
  
\begin{lstlisting}[frame=single]  
IF...ELSEIF...ELSE			# Conditional
CASE...WHEN...ELSE			# Conditional
WHILE...DO...LOOP			# Repetition
REPEAT...UNTIL...END REPEAT	# Repetition
DECLARE CURSOR FOR			# Result set for looping
DECLARE...HANDLER			# Error handler
\end{lstlisting} 




  
  
  
\subsection*{IF Statements}
  
 \begin{lstlisting}[frame=single]  
DELIMITER $$

CREATE PROCEDURE procIF()

BEGIN
	DECLARE flight_date DATE;
	
	SELECT date INTO flight_date FROM Flights WHERE fid = 1;
	
	IF flight_date > NOW() THEN
		SELECT 'Missed your flight.  Sorry';
	END IF
END$$

DELIMITER ;
\end{lstlisting} 




  
  
\subsection*{CASE Statements}
  
\begin{lstlisting}[frame=single]  
DELIMITER $$

CREATE PROCEDURE procLOOP()

BEGIN
	DECLARE airline CHAR(2);
	
	SELECT carrier INTO airline FROM Flights WHERE fid = 1;
	
	CASE airline
		WHEN 'AA' THEN 
			SELECT 'You fly American Airlines';
		WHEN 'UA' THEN
			SELECT 'You fly United Airlines';
		ELSE
			SELECT 'You fly some other airlines';
		END CASE;
END$$

DELIMITER ;
\end{lstlisting} 


  
  
  
  
  
  
\subsection*{Repetition Structures (Looping)}
There are three looping structures: \texttt{WHILE}, \texttt{REPEAT}, and just \texttt{LOOP}.

% WHILE
\subsection*{WHILE Loop}
\begin{lstlisting}[frame=single]  
DELIMITER $$

CREATE PROCEDURE procWHILE()

BEGIN
	DECLARE i INT DEFAULT 1;
	DECLARE s VARCHAR(45) DEFAULT ' ';

	WHILE i < 10 DO
		SET s = CONCAT_WS(',', s, i);
		SET i = i +  1;
	END WHILE;
	
	SELECT s AS Result;
END$$

DELIMITER ;
\end{lstlisting} 







% REPEAT
\subsection*{REPEAT Loop}
\begin{lstlisting}[frame=single]  
DELIMITER $$

CREATE PROCEDURE procREPEAT()

BEGIN
	DECLARE i INT DEFAULT 1;
	DECLARE s VARCHAR(45) DEFAULT ' ';

	REPEAT
		SET s = CONCAT_WS(',', s, i);
		SET i = i +  1;
	UNTIL i = 10
	END REPEAT;
	
	SELECT s AS Result;
END$$

DELIMITER ;
\end{lstlisting} 

  
  
  
  
 \subsection*{CURSOR}
 A cursor is a row-based operation to obtain subset of data by looping through each row one at a time.  The steps to work with cursors are:
 
 \begin{outline}[enumerate]
 	\1 DECLARE cursor 	- \texttt{DECLARE cursorname CURSOR FOR}, then specify the \texttt{SELECT} statement.  
	\1 OPEN cursor	 	- Opens cursor
	\1 FETCH cursor	- fetch row from the cursor into a program variable
	\1 CLOSE cursor 	- when operations are complete
 \end{outline}
 
  
\begin{lstlisting}[frame=single]  

DELIMITER $$
USE `dbsoln_Store`$$
CREATE PROCEDURE `UpdateInventory` 
(
       IN oid 	int		
)
   BEGIN
	declare xProduct INT;
	declare xInventory INT;
    declare row_not_found tinyint default false;
	declare update_count INT default 0;
    
	Declare curInventory CURSOR FOR
		select D.product, D.qty from OrderDetails D where `order`=oid;
	
    Declare continue handler for not found
		SET row_not_found = TRUE;
        
	OPEN curInventory;
    
    while row_not_found = FALSE DO
		fetch curInventory into xProduct, xInventory;
        update Products set inventory = inventory - xInventory
        where product_id = xProduct;
        set update_count = update_count + 1;
	end while;
	
    close curInventory;

	select concat(update_count, ' row(s) updated.');

	END$$

DELIMITER ;

# Test 
# call UpdateInventory(2)
\end{lstlisting} 

%+Bibliography
%\begin{thebibliography}{99}
%\bibitem{Label1} ...
%\bibitem{Label2} ...
%\end{thebibliography}
%-Bibliography



 

\end{document}
