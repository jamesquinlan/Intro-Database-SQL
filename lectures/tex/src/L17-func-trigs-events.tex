% &latex

% Wed Apr  23 02:54:19 EDT 2021
%+++++++++++++++++++++++++++++++++++++++++++++++++++++++++++++++++++++
% SUMMARY   :  Introduction to Database
%           		:  User functions, triggers, and events
%			:
%            		:  (c) 2021. Quinlan, J 
%            		:  MIT License.
%+++++++++++++++++++++++++++++++++++++++++++++++++++++++++++++++++++++

\documentclass{article}

\newtheorem{definition}{Definition}
\newtheorem{example}{Example}


\usepackage{outlines}
\usepackage{enumitem}
\setenumerate[1]{label=\arabic*.}
\setenumerate[2]{label=\alph*.}
\setenumerate[3]{label=\roman*.}
\setenumerate[4]{label=\alph*.}

\usepackage{graphicx}
\DeclareGraphicsRule{.tif}{png}{.png}{`convert #1 `dirname #1`/`basename #1 .tif`.png}

% CODE (SQL)

\usepackage{framed}  % https://latexcolor.com
\usepackage{xcolor}


\usepackage{tcolorbox, graphicx}
\usepackage{colortbl}
\usepackage{textcomp}

% Colors
%\definecolor{anti-flashwhite}{rgb}{0.95, 0.95, 0.96}
%\definecolor{magnolia}{rgb}{0.97, 0.96, 1.0}
\definecolor{antiquewhite}{rgb}{0.96, 0.96, 0.96} % {0.98, 0.92, 0.84}
%\definecolor{shadecolor}{rgb}{0.95,0.975,0.997}
%\definecolor{shadecode}{rgb}{0.91,0.91,0.91}
%\definecolor{red_1}{rgb}{1,0.8,0.8}
%\definecolor{yellow_1}{rgb}{1,0.96,0.63}
%\definecolor{orange}{rgb}{1,0.5,0}
\definecolor{appleGray}{rgb}{0.75,0.75,0.75}
\definecolor{lightGray}{rgb}{0.975,0.975,0.975}
\definecolor{relhead}{rgb}{0.70,0.80,0.90}
\definecolor{borderGray}{rgb}{0.8,0.8,0.8}
%\definecolor{gray}{rgb}{0.975,0.975,0.975}
\definecolor{nearwhite}{rgb}{0.985,0.985,0.985}
%\definecolor{supergray}{cmyk}{0,0,0.04,0}
%\definecolor{stainlessSteel}{cmyk}{0,0,0.02,0.12}
%\definecolor{mygreen}{rgb}{0,0.6,0}
\definecolor{mygray}{rgb}{0.20,0.20,0.20}
\definecolor{mymauve}{rgb}{0.58,0,0.82}
\definecolor{excel}{rgb}{0.94, 0.94, 0.94}



\usepackage{listings}  

 \lstset{ 
  backgroundcolor=\color{white},  	 % background color; e.g., nearwhite you must add \usepackage{color} or \usepackage{xcolor}; should come as last argument
  basicstyle=\footnotesize\ttfamily,        % the size of the fonts that are used for the code
  breakatwhitespace=false,         		    % sets if automatic breaks should only happen at whitespace
  breaklines=true,                 		    % sets automatic line breaking
  framextopmargin=5pt,
  framexleftmargin=5pt, 
  framexbottommargin=5pt,
  framexrightmargin=0pt,
  framesep=0pt,
  captionpos=b,                    			% sets the caption-position to bottom
  commentstyle=\color{mygreen},    		    % comment style
  morecomment=[s]{/*}{*/},
  deletekeywords={...},            			% if want to delete keywords from the given language
  escapeinside={\%*}{*)},          			% if you want to add LaTeX within your code
  extendedchars=true,              			% lets you use non-ASCII characters; for 8-bits encodings only, does not work with UTF-8
  frame=single,	                   			% adds a frame around the code
  keepspaces=false,               			% keeps spaces in text, useful for keeping indentation of code (possibly needs columns=flexible)
  keywordstyle=\color{blue},      		% keyword style blue
  language=SQL,                 			% the language of the code
  morekeywords={ORDER, USE, DELIMITER, CALL, DECLARE, IF, ELSE, ELSEIF, WHILE, DO, LOOP, REPEAT, UNTIL, CURSOR, FOR, HANDLER, OUT, INTO, FROM, RETURNS, RETURN, FUNCTION, SHOW, TRIGGER, TRIGGERS, END$$, EVENT},   % if you want to add more keywords to the set
  numbers=none,                    			% where to put the line-numbers; possible values are (none, left, right)
  numbersep=0pt,                   			% how far the line-numbers are from the code
  numberstyle=\tiny\color{mygray}, 		    % the style that is used for the line-numbers
  rulecolor=\color{appleGray},  % appleGray     % if not set, the frame-color may be changed on line-breaks within not-black text (e.g. comments (green here))
  sensitive=true,
  showspaces=false,                			% show spaces everywhere adding particular underscores; it overrides 'showstringspaces'
  showstringspaces=false,          		    % underline spaces within strings only
  showtabs=false,                  			% show tabs within strings adding particular underscores
  stepnumber=2,                    			% the step between two line-numbers. If it's 1, each line will be numbered
  stringstyle=\color{mymauve},     		    % string literal style
  tabsize=4,	                   			% sets default tabsize to 2 spaces
  title=\lstname,                  			% show the filename of files included with \lstinputlisting; also try caption instead of title
  upquote=true,      % Straight quotes
  belowcaptionskip=0em,
  belowskip=0em
}
 



\begin{document}

%+Title
\title{Functions, Triggers, and Events}
\author{DSC 301: Lecture 17}
\date{\today}
\maketitle
%-Title

%+Abstract
\begin{abstract}
\noindent MySQL 5.0 introduced stored procedures, functions, triggers, and events.  In this lecture we cover these three stored programs.  
\end{abstract}
%-Abstract

%+Contents
% \tableofcontents
%-Contents

% ------------------------------------------------------------





% --------- Lecture Objectives ------- %
\section*{Lecture Objectives}
We cover the following three topics in this lesson:
\begin{outline}
        \1  User defined functions
        \1  Triggers
        \1  Events        
\end{outline}










% --------- SECTION: User defined functions ------- %
\section{User Defined Functions}
 
Aggregate functions such as \texttt{sum}, \texttt{avg}, \texttt{count} and string functions including \texttt{concat}, \texttt{ltrim}, and \texttt{substring} are built in functions.  Users can create their own utility functions.  These are called \textbf{user defined functions}, or simply \textbf{functions}.  \\

\noindent Before we present a function template, here are a few items to keep in mind when working with functions in MySQL.


\subsubsection*{Function Rules}
\noindent Functions, 
\begin{outline}
        \1  return a single value, as opposed to a result set (therefore functions are sometimes called \textit{scalar functions})
        \1   cannot execute \texttt{UPDATE}, \texttt{INSERT}, or \texttt{DELETE} statements.  
	\1  are called by using it as an expression just like built-in functions (e.g., \texttt{COUNT})
\end{outline}


% TEMPLATE:  user defined function
\subsubsection*{Function Template}

\begin{lstlisting}[frame=single]  
DELIMITER $$

CREATE FUNCTION name_of_function
	(
	[parameter1 dataType]
	[, parameter2 dataType, ...]
	)
RETURNS dataType

BEGIN
	DECLARE local_variable dataType;
	
	# SQL statement(s) and/or other code
	
	RETURN(local_variable);
	
END$$

DELIMITER ;
\end{lstlisting} 



% EXAMPLE: user defined function
\begin{example}
Write a function that accepts an order number and returns the total.  Then apply this function to order number 1.
\end{example}

\begin{lstlisting}[frame=single]  
use dbsoln_Store;
DROP FUNCTION if exists getTotal;
DELIMITER $$

CREATE FUNCTION getTotal
	(
		oid 	INT
	)
RETURNS decimal(9,2)

BEGIN
	DECLARE total decimal(9,3);
	
	SELECT sum(qty*msrp) INTO total 
    FROM OrderDetails D 
	INNER JOIN
        Products P ON D.product = P.product_id 
    WHERE D.order = oid;
	
	RETURN(total);	
END$$

DELIMITER ;

#  Function in use:
SELECT getTotal(1);
\end{lstlisting} 
% --------------------------------------------------





\begin{example}
	Write a function that doubles MSRP and then adds \$10.00.  
\end{example}

\begin{lstlisting}[frame=single]  
USE dbsoln_Store;
DROP function IF EXISTS qpricing;

DELIMITER $$
CREATE FUNCTION  qpricing(x decimal(9,2))
RETURNS decimal(9,2)

BEGIN
	DECLARE y int;
        SET y = 2*x+10;
	RETURN y;
END$$

DELIMITER ;
\end{lstlisting} 


\begin{lstlisting}[frame=single]  
# Test the function:
SELECT qpricing(msrp) FROM Products WHERE product_id = 1111;
SELECT qpricing(msrp) FROM Products WHERE product_id in(1111,1345,1853);
\end{lstlisting} 




% HOW TO DROP A FUNCTION (IF EXISTS)
\subsubsection*{DROP a function}

To drop a function type:  
\begin{lstlisting}[frame=single]  
DROP FUNCTION [IF EXISTS] function_name
\end{lstlisting} 

% --------------------------------------------------
\hspace{-0.5cm}\rule[-0.101in]{\textwidth}{0.0025in}
% --------------------------------------------------











% --------- SECTION: Triggers ------- %
\section{Triggers}
 
\textbf{Triggers} are classified as \textit{stored programs}.  Triggers can be executed before or after an \texttt{INSERT}, \texttt{UPDATE}, or \texttt{DELETE} query is executed.  Triggers aid in data consistency.   A \texttt{FOR EACH ROW} clause must be specified that indicates that the statements are executed for all modified rows.  If the body executes a single statement, the \texttt{BEGIN} and \texttt{END} statements are not needed.  \texttt{NEW} and \texttt{OLD} keywords indicate whether we are working with new or old values.  NOTE: If trigger uses \texttt{INSERT}, then \texttt{OLD} cannot be used as there are no old values.  Likewise, \texttt{NEW} cannot be used with \texttt{DELETE}, since there will not be a new row deleted (you only delete old rows).\\

\newpage


% TEMPLATE: Triggers
\noindent The basic syntax for triggers is:

\begin{lstlisting}[frame=single]  
DELIMITER $$

CREATE TRIGGER name_of_trigger
	{BEFORE | AFTER} {INSERT | UPDATE | DELETE} ON Table_Name
	FOR EACH ROW

BEGIN
	# SQL statement(s)	
END$$
\end{lstlisting} 




% EXAMPLE: create a trigger
\begin{example}
Create a trigger to capitalize the first letter of a new customers first and last name.  Note how this adds consistency.  
\end{example}

\begin{lstlisting}[frame=single]  
use dbsoln_Store;

DELIMITER $$
DROP TRIGGER IF EXISTS `dbsoln_Store`.`capFname` $$
CREATE TRIGGER `dbsoln_Store`.`capFname`
	BEFORE INSERT ON `Customers`
	FOR EACH ROW

BEGIN
	SET NEW.fname = concat(UPPER(LEFT(NEW.fname,1)),LOWER(SUBSTRING(NEW.fname,2,LENGTH(NEW.fname))));

END$$

DELIMITER ;
\end{lstlisting} 
% --------------------------------------------------




% How to SHOW and DROP triggers 
\subsubsection*{SHOW and DROP Triggers}

You can show a list of all triggers using: 

% show
\begin{lstlisting}[frame=none]  
SHOW TRIGGERS IN database;
\end{lstlisting} 

\noindent The \texttt{LIKE} keyword can be added to the SQL statement to filter the triggers, 

\begin{lstlisting}[frame=none]  
SHOW TRIGGERS IN database LIKE 'cap%';
\end{lstlisting} 




% drop
\noindent Dropping a trigger is done by, 
\begin{lstlisting}[frame=none]  
DROP TRIGGER [IF EXISTS] trigger_name;
\end{lstlisting} 

\noindent Note: \texttt{IF EXISTS} is optional.

% --------------------------------------------------
\hspace{-0.5cm}\rule[-0.101in]{\textwidth}{0.0025in}
% --------------------------------------------------




 
















% --------- SECTION: Events  ------- %
\section{Events}
 
 \textbf{Events} (introduced in MySQL 5.1) are executed on a schedule. They provide automation of tasks that need to be run  \textit{one-time} or \textit{recurring}.    

\begin{outline}
	\1 To use events, the \textit{event scheduler} must be turned \texttt{ON}.  By default it is \texttt{OFF}.
		
		\2 To see if the event schedule is on, type: \texttt{SHOW VARIABLES LIKE 'event\%';}
		\2 If \texttt{OFF}, then \texttt{SET GLOBAL event\_scheduler $=$ ON;}
		\2 Note: 1 and 0 are equivalent to \texttt{ON} and \texttt{OFF} respectively
		
	% \1 An event can be \textit{one-time} or \textit{recurring}
	
\end{outline}

 
 % TEMPATE: Event
\begin{lstlisting}[frame=single]  
USE whichDatabase;

# Change the delimiter because semi-colon used inside
DELIMITER $$

CREATE EVENT name_of_event
ON SCHEDULE 
{AT timestamp  |  EVERY interval [STARTS timestamp]  [ENDS timestamp]
DO 
BEGIN
	# SQL Statements
END$$

# Change delimiter back to semicolon
DELIMITER ;
\end{lstlisting} 






\begin{lstlisting}[frame=single]  
DROP EVENT [IF EXISTS] event_name
\end{lstlisting} 

\noindent For example, delete certain records monthly.

\begin{lstlisting}[frame=single]  
USE whichDatabase;
DELIMITER $$

CREATE EVENT outofstock
ON SCHEDULE 
EVERY 1 month STARTS '2021-06-01'
DO 
BEGIN
	DELETE FROM Products 
	WHERE inventory < 1;
END$$

# Change delimiter back to semicolon
DELIMITER ;
\end{lstlisting} 


% select product_id, name, inventory, price, msrp, qbomb(msrp) as QBOMB from Products; 


% --------------------------------------------------
\hspace{-0.5cm}\rule[-0.101in]{\textwidth}{0.0025in}
% --------------------------------------------------



  
%+Bibliography
%\begin{thebibliography}{99}
%\bibitem{Label1} ...
%\bibitem{Label2} ...
%\end{thebibliography}
%-Bibliography



 

\end{document}

% Outline environment templates
% ---------------- %
\begin{outline}[enumerate]

\end{outline}
% ---------------- %
\begin{outline}
        
\end{outline}