%&latex
\documentclass{article}

 
\usepackage{amsthm}
\newtheorem{definition}{Definition}
\newtheorem{example}{Example}


\usepackage{outlines}
\usepackage{amssymb}
\usepackage{enumitem}
\setenumerate[1]{label=\arabic*.}
\setenumerate[2]{label=\alph*.}
\setenumerate[3]{label=\roman*.}
\setenumerate[4]{label=\alph*.}

\usepackage{graphicx}
\DeclareGraphicsRule{.tif}{png}{.png}{`convert #1 `dirname #1`/`basename #1 .tif`.png}



\usepackage{framed}  % https://latexcolor.com
\usepackage{xcolor}


\usepackage{tcolorbox, graphicx}
\usepackage{colortbl}
\usepackage{textcomp}

% Colors
%\definecolor{anti-flashwhite}{rgb}{0.95, 0.95, 0.96}
%\definecolor{magnolia}{rgb}{0.97, 0.96, 1.0}
\definecolor{antiquewhite}{rgb}{0.96, 0.96, 0.96} % {0.98, 0.92, 0.84}
%\definecolor{shadecolor}{rgb}{0.95,0.975,0.997}
%\definecolor{shadecode}{rgb}{0.91,0.91,0.91}
%\definecolor{red_1}{rgb}{1,0.8,0.8}
%\definecolor{yellow_1}{rgb}{1,0.96,0.63}
%\definecolor{orange}{rgb}{1,0.5,0}
\definecolor{appleGray}{rgb}{0.75,0.75,0.75}
\definecolor{lightGray}{rgb}{0.975,0.975,0.975}
\definecolor{relhead}{rgb}{0.70,0.80,0.90}
\definecolor{borderGray}{rgb}{0.8,0.8,0.8}
%\definecolor{gray}{rgb}{0.975,0.975,0.975}
\definecolor{nearwhite}{rgb}{0.985,0.985,0.985}
%\definecolor{supergray}{cmyk}{0,0,0.04,0}
%\definecolor{stainlessSteel}{cmyk}{0,0,0.02,0.12}
%\definecolor{mygreen}{rgb}{0,0.6,0}
\definecolor{mygray}{rgb}{0.20,0.20,0.20}
\definecolor{mymauve}{rgb}{0.58,0,0.82}
\definecolor{excel}{rgb}{0.94, 0.94, 0.94}



\usepackage{listings}  

 \lstset{ 
  backgroundcolor=\color{white},  	 % background color; e.g., nearwhite you must add \usepackage{color} or \usepackage{xcolor}; should come as last argument
  basicstyle=\footnotesize\ttfamily,        % the size of the fonts that are used for the code
  breakatwhitespace=false,         		    % sets if automatic breaks should only happen at whitespace
  breaklines=true,                 		    % sets automatic line breaking
  framextopmargin=5pt,
  framexleftmargin=5pt, 
  framexbottommargin=5pt,
  framexrightmargin=0pt,
  framesep=0pt,
  captionpos=b,                    			% sets the caption-position to bottom
  commentstyle=\color{mygreen},    		    % comment style
  morecomment=[s]{/*}{*/},
  deletekeywords={...},            			% if want to delete keywords from the given language
  escapeinside={\%*}{*)},          			% if you want to add LaTeX within your code
  extendedchars=true,              			% lets you use non-ASCII characters; for 8-bits encodings only, does not work with UTF-8
  frame=single,	                   			% adds a frame around the code
  keepspaces=false,               			% keeps spaces in text, useful for keeping indentation of code (possibly needs columns=flexible)
  keywordstyle=\color{blue},      		% keyword style blue
  language=SQL,                 			% the language of the code
  morekeywords={ORDER, USE, DELIMITER, CALL, DECLARE, IF, ELSE, ELSEIF, WHILE, DO, LOOP, REPEAT, UNTIL, CURSOR, FOR, HANDLER, OUT, INTO, FROM, RETURNS, RETURN, FUNCTION, SHOW, TRIGGER, TRIGGERS, END$$, EVENT},   % if you want to add more keywords to the set
  numbers=none,                    			% where to put the line-numbers; possible values are (none, left, right)
  numbersep=0pt,                   			% how far the line-numbers are from the code
  numberstyle=\tiny\color{mygray}, 		    % the style that is used for the line-numbers
  rulecolor=\color{appleGray},  % appleGray     % if not set, the frame-color may be changed on line-breaks within not-black text (e.g. comments (green here))
  sensitive=true,
  showspaces=false,                			% show spaces everywhere adding particular underscores; it overrides 'showstringspaces'
  showstringspaces=false,          		    % underline spaces within strings only
  showtabs=false,                  			% show tabs within strings adding particular underscores
  stepnumber=2,                    			% the step between two line-numbers. If it's 1, each line will be numbered
  stringstyle=\color{mymauve},     		    % string literal style
  tabsize=4,	                   			% sets default tabsize to 2 spaces
  title=\lstname,                  			% show the filename of files included with \lstinputlisting; also try caption instead of title
  upquote=true,      % Straight quotes
  belowcaptionskip=0em,
  belowskip=0em
}
 




\begin{document}

%+Title
\title{Relational Algebra II}
\author{DSC 301: Lecture 11}
\date{March 12, 2021} % \today
\maketitle
%-Title

%+Abstract
%\begin{abstract}
%    There is abstract text that you should replace with your own. 
%\end{abstract}
%-Abstract



%+Contents
% \tableofcontents
%-Contents

% ---------------- %
\begin{outline}[enumerate]

\end{outline}
% ---------------- %
\begin{outline}
        
\end{outline}




% --------- Lecture Objectives ------- %
\section*{Lecture Objectives}
\begin{outline}
        \1 Review basic relational algebra
        \1  Advanced (relational) algebra        
        \1  Introduce JOINS
        
       
       
        

\end{outline}

% --------------------------------------------------
\hspace{-0.5cm}\rule[-0.101in]{\textwidth}{0.0025in}
% --------------------------------------------------
% 
% 
%\begin{definition}
%asdfasdf
%\end{definition}









% --------   -------- %
\section*{Relational Algebra}
 
Recall from Lecture 3 that a \textit{relation} (a.k.a, table, tuples) 
is a subset of a Cartesian product and a \textit{relational algebra} defines a set of operations on relations.  An \underline{essential} property of all algebras  is the \textbf{closure} property, which says that two elements from the set combined (in some way) will remain within the set.  Symbolically, if $a,b \in R$, then $a \odot b \in R$ where $\odot$ is some operation (e.g., addition, multiplication, etc.).          


\subsection*{Basic Operations on Relational Algebras}

\begin{outline}

 
    
    
   

    
\1  \textbf{Selection}: A subset of rows, $\sigma_p(R)$, where $p$ is a predicate and $R$ a relation. 
       
       
             %   \[
             %   \sigma_p(R)  = \{t \; | \; t \in R \; \wedge \; p(t) 
             %   \textit{ true } \}
             %   \]
   
       
 

\begin{lstlisting}[frame=single]  
SELECT * FROM Flights where distance > 1000;
\end{lstlisting} 
%
%      
        
         


   
   
   
   \1  \textbf{Projection}: A Subset of columns, symbolically, 
        $\Pi_{A_1, A_2, \dots, A_k}(R)$
               
        

\begin{lstlisting}[frame=single]  
SELECT carrier, dest FROM Flights;
\end{lstlisting} 
%
%      
    
               
         

\1 \textbf{Rename}: $\rho_S(R) \to S$, where we rename relation $R$ to $S$.

Also, $\rho_R_{(A_1, A_2, \dots , A_n)}(R)$ renames the attributes of $R$
to $A_1, A_2, \dots , A_n$.  



\begin{lstlisting}[frame=single]  
SELECT dest as City FROM Flights;
\end{lstlisting} 
%
%
        
\end{outline} 


 

%\begin{lstlisting}[frame=single]  
%SELECT COUNT(<attribute>) FROM <TableX>;
%\end{lstlisting} 


% --------------------------------------------------
\hspace{-0.5cm}\rule[-0.101in]{\textwidth}{0.0025in}
% --------------------------------------------------
  
  
  


 \begin{figure}[htbp] %  figure placement: here, top, bottom, or page
    \centering
    \includegraphics[width=2in]{Selection.png} 
    \caption{Selection}
    \label{fig:selection}
 \end{figure}



  
  

 \begin{figure}[htbp] %  figure placement: here, top, bottom, or page
    \centering
    \includegraphics[width=2in]{Projection.png} 
    \caption{Projection}
    \label{fig:projection}
 \end{figure}





% --------------------------------------------------
\hspace{-0.5cm}\rule[-0.101in]{\textwidth}{0.0025in}
% --------------------------------------------------
  
     








\subsection*{Join Operations}


A\ \textbf{join} combines two or more tables to retreive data from multiple tables.  Tables included in the join are listed in the \texttt{FROM} clause.  The \texttt{JOIN} itself is contained in the \texttt{WHERE} clause.   The common operators that can be used to join tables include: $=, <, >, \le, \ge, <>, !=$, \texttt{BETWEEN}, \texttt{LIKE}, and \texttt{NOT}.  However, most frequently the equal sign ($=$) is used.          The common joins are:

\begin{outline}
        \1 Inner joins, $R \bowtie S$        
                
                
        \1 Outer-joins, $R \ltimes S$ or 
                \2 Left - Returns all records from the left table, and the matched records from the right table
                \2 Right - Returns all records from the right table, and the matched records from the left table
                \2 Full - Returns all records when there is a match in either left or right table
\end{outline}

        \1 Semi-join - (half join)
        
  
  
  
  
 /*
DSC301 Spring 2021
University of New England
Quinlan, J.

Topic: Math Functions

*/

USE dbsoln_Seattle2014;

SELECT 
    distinct F1.carrier, F1.dest
FROM
    Flights AS F1
        JOIN
    Flights AS F2 ON F1.dest = F2.dest
        AND F1.carrier <> F2.carrier LIMIT 500; 



      
        
\[

%\bowtie

% \ltimes
 
 %\rtimes
 
% \triangleleft
 
 
\]


 
  
 \subsubsection*{Inner Join}
  
  \begin{outline}
        \1 Inner joins
                \2 Returns records that have matching values in both tables
                \2 Most frequent
                \2 Also called \textit{equijoin}
    \end{outline}
  
                  

\begin{lstlisting}[frame=single]  
SELECT 
        Table1.column1, Table2.column2 
FROM 
        Table1 INNER JOIN Table2
ON 
        Table1.somecol = Table2.samecol;
\end{lstlisting} 
%
%      
  \textbf{Example}: Note the table aliases.          
                

\begin{lstlisting}[frame=single]  
SELECT 
    O.id, O.date, C.fname, C.lname
FROM
    Orders AS O
        INNER JOIN
    Customers AS C ON O.customer = C.id
ORDER BY O.date; 
\end{lstlisting} 
%
%              


         
\begin{lstlisting}[frame=single]  
SELECT 
        Table1.column1, Table2.column2 
FROM 
        Table1, Table2
WHERE 
        Table1.somecol = Table2.samecol;
\end{lstlisting} 
%
%      
  \textbf{Example}: Note the table aliases.          
                

\begin{lstlisting}[frame=single]  
SELECT 
    O.id, O.date, C.fname, C.lname
FROM
    Orders AS O,
    Customers AS C
WHERE
    O.customer = C.id
ORDER BY O.date;
\end{lstlisting} 
%
%                




















\subsubsection*{Outer Joins}

\begin{outline}
     
                
        \1 Outer-joins
                \2 Left - Returns all records from the left table, and the
matched records from the right table
                \2 Right - Returns all records from the right table, and
the matched records from the left table
                \2 Full - Returns all records when there is a match in either
left or right table
\end{outline}


  
  
%+Bibliography
%\begin{thebibliography}{99}
%\bibitem{Label1} ...
%\bibitem{Label2} ...
%\end{thebibliography}
%-Bibliography



 

\end{document}
